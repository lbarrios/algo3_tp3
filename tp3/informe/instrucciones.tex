
Para compilar el proyecto completa: entrar en la carpeta src y correr make all.

Para compilar cada una de las partes por separado: en la carpeta src, estan las carpetas backtracking, grasp, greedy\_heuristic\_\{A,B,C\} y local\_search. Se puede compilar cada parte independiente, entrando en cada carpeta y ejecuenta make. 

Archivo MakeFile:

\begin{itemize}
  \item Hay un archivo Makefile.common que se usa como input para los makefile del resto de las carpetas.
  La sintaxis para incluirlo es: \texttt{include ../Makefile.common}
  
  \item Por defecto, compila el archivo que tenga el mismo nombre de la carpeta, y busca el .cpp y el .h.
	  Ejemplo, si estamos en la carpeta backtracking va a buscar backtracking.cpp y backtracking.h y lo compila en backtracking.
	  
  \item Siempre va a crear una carpeta OBJS en donde guarda los archivos .o
  
  \item Cuando se desean incluir archivos que se encuentren en la carpeta common, se debe anteponer COMMON\_OBJS al include:
  
  \begin{verbatim}
      COMMON\_OBJS := ClassName1 ClassName2
      include ../Makefile.common
  \end{verbatim}
  
  en donde ClassName es el nombre de la clase. Ejemplo, COMMON\_OBJS := Graph Edge

  \item Pueden agregar targets específicos que deseen luego del include ../Makefile.common, tal y como harían en un Makefile normal
  
\end{itemize}