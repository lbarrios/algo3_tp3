Para compilar el proyecto completo: entrar en la carpeta \texttt{src} y correr \texttt{make all}.

Para compilar cada una de las partes por separado: en la carpeta \texttt{src}, estan las carpetas \texttt{backtracking}, \texttt{grasp}, \texttt{greedy\_heuristic\_\{A,B,C\}} y \texttt{local\_search}. Se puede compilar cada parte de forma independiente, entrando en cada carpeta y ejecutando \texttt{make}. 

Funcionamiento de los archivos \texttt{Makefile}:

\begin{itemize}
    \item Hay un archivo \texttt{Makefile.common} que se usa como input para los \texttt{Makefile} del resto de las carpetas.
  La sintaxis para incluirlo es: \texttt{include ../Makefile.common}
  
  \item Por defecto, compila el archivo que tenga el mismo nombre de la carpeta, y busca el \texttt{.cpp} y el \texttt{.h}.
      Por ejemplo, si estamos en la carpeta \texttt{backtracking} va a buscar \texttt{backtracking.cpp} y \texttt{backtracking.h} y lo va a compilar en \texttt{backtracking}.
	  
  \item Siempre va a crear una carpeta \texttt{OBJS} en donde guarda los archivos \texttt{.o}
  
  \item Cuando se desean incluir archivos que se encuentren en la carpeta \texttt{common}, se debe anteponer \texttt{COMMON\_OBJS} al \texttt{include}:
  
  \begin{verbatim}
      COMMON\_OBJS := ClassName1 ClassName2
      include ../Makefile.common
  \end{verbatim}
  
  en donde \texttt{ClassName} es el nombre de la clase. Por ejemplo, \texttt{COMMON\_OBJS := Graph Edge}

  \item Se pueden agregar targets específicos que luego del \texttt{include ../Makefile.common}.
  
\end{itemize}
