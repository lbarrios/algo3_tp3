% ------ headers globales -------------
\documentclass[12pt, a4paper, twoside]{article}
\usepackage{header_tp3}
\raggedbottom % para evitar que se formatee automaticamente "vertically justified" 

%uso: \ponerGrafico{file}{caption}{scale}{label}
\newcommand{\ponerGrafico}[4]
{\begin{figure}[H]
	\centering
	\subfloat{\includegraphics[scale=#3]{#1}}
	\caption{#2} \label{fig:#4}
\end{figure}
}

%\includeonly{instrucciones, codigo-fuente, conclusiones, introduccion, problema1, problema2, problema3}
\begin{document}{}
% -------------------------------------
% -- Carátula --
\newpage{\pagestyle{empty}% parametros para la caratula (caratula.sty)

\materia{Algoritmos y Estructuras de Datos III}
\submateria{Entrega de TP}
\titulo{Trabajo Práctico 2}
\subtitulo{Técnicas Algorítmicas Avanzadas}
\fecha{Viernes 9 de Mayo de 2014}
\integrante{Barrios, Leandro E.}{404/11}{ezequiel.barrios@gmail.com}
\integrante{Benegas, Gonzalo}{958/12}{gsbenegas@gmail.com}
\integrante{Duarte, Miguel}{904/11}{miguelfeliped@gmail.com}
\integrante{Niikado, Marina}{711/07}{mariniik@yahoo.com.ar}
\grupo{Grupo \red{\texttt{$(-1111111111111111111111111111111111111111111111111111111111111001_{2})$}}}

\maketitle\cleardoublepage}
  % -- nota:
  %       si no esta dentro de \newpage{} se rompe la numeracion y el encabezadp
  %       \cleardoublepage es para que el indice no se imprima atras de la caratula
%~ \pagestyle{empty}% parametros para la caratula (caratula.sty)

\materia{Algoritmos y Estructuras de Datos III}
\submateria{Entrega de TP}
\titulo{Trabajo Práctico 2}
\subtitulo{Técnicas Algorítmicas Avanzadas}
\fecha{Viernes 9 de Mayo de 2014}
\integrante{Barrios, Leandro E.}{404/11}{ezequiel.barrios@gmail.com}
\integrante{Benegas, Gonzalo}{958/12}{gsbenegas@gmail.com}
\integrante{Duarte, Miguel}{904/11}{miguelfeliped@gmail.com}
\integrante{Niikado, Marina}{711/07}{mariniik@yahoo.com.ar}
\grupo{Grupo \red{\texttt{$(-1111111111111111111111111111111111111111111111111111111111111001_{2})$}}}

\maketitle 
\setcounter{page}{1}

%-- Índice --
\newpage{\pagestyle{empty}\tableofcontents\cleardoublepage}

%-- Dentro de TP3 redefino ciertos comandos para que se pueda compilar todo individualmente --
\begin{TP3}
%-- Introduccion --
\section{Introducción}\label{sec:introduccion}

En este trabajo pr\'actico nos piden analizar el problema del \textit{Camino Acotado de Costo M\'inimo} ($CACM$), y desarrollar distintos algoritmos para resolverlo. 

Dado un Grafo $G = (V,E)$, dos funciones de peso $\omega_1, \omega_2: V \mapsto \mathbb{R}_+$, un natural $K$ y dos nodos $u,v \in V$, el problema consiste en encontrar, entre todos los caminos $P$ entre $u$ y $v$ que cumplen $\omega_1(P) \leq K$, el que minimice $\omega_2(P)$. 

Donde si $w$ es una función de peso definida sobre aristas, se entiende $w(P)$ como 

\begin{equation*}
\sum_{e \text{ arista de } P} w(e)
\end{equation*}

$CACM$ es un problema conocido, y tiene muchas aplicaciones en la vida real. Una agencia de vuelos puede estar interesada en ofrecer el viaje más corto entre dos ciudades, dado un cliente con un presupuesto acotado. Los nodos representan ciudades. $u$ y $v$ son las ciudades de origen y destino, respectivamente. Una arista es un vuelo particular entre dos ciudades. El peso por $\omega_1$ es el costo del pasaje y el peso por $\omega_2$  es la duración del vuelo. Un camino es una secuencia de vuelos, es decir, un vuelo que puede o no tener escalas. K viene a ser el presupuesto del cliente. El camino buscado es entonces el que, entre todos los vuelos (con o sin escalas) entre la ciudad de origen y destino que el cliente puede pagar, tiene la menor duración.

Otro ejemplo similar está relacionado con el Mapa Interactivo de la Ciudad de Buenos Aires. Se busca llegar de un punto de la ciudad a otro en el
menor tiempo, aunque se puede especificar la máxima cantidad de metros por caminar. Los nodos son alturas de calles, es decir puntos geográficos sobre alguna calle. Una arista entre dos nodos es un segmento de calle - junto con un medio de transporte - que une dos puntos geográficos. El peso por $\omega_1$ es la longitud del segmento si el medio de transporte es `caminar' y 0 si no. El peso por $\omega_2$ es el tiempo que demora recorrer el segmento usando el correspondiente medio de transporte. El valor K, especificado por el usuario, es la máxima cantidad de metros que está dispuesto a caminar. Un camino entre un punto $u$ de origen a otro punto $v$ de destino es una sucesión de segmentos recorridos con un correspondiente medio de transporte. Se busca, entre todos los caminos cuyos metros caminados totales no exceden K, el de menor duración.

%Un ejemplo reciente fue obtenido del \textit{MIT Technology Review}\footnote{technologyreview.com/view/528836/forget-the-shortest-route-across-a-city-new-algorithm-finds-the-most-beautiful/}. \fixme{guarda que este es maximo, explicar}

Aunque $CACM$ es un problema conocido, no se conocen algoritmos polinomiales que lo resuelvan y se estima que pertenece al conjunto de problemas NP. En este trabajo se analizaran varios métodos para resolverlo: una soluci\'on exacta y 5 aproximaciones a trav\'es de heur\'isticas. En concreto, se implementarán los siguientes algoritmos:

\begin{enumerate}
\item Un \textit{Backtracking} como algoritmo exacto.
\item Una heur\'istica \textit{constructiva greedy}.
\item Una heur\'istica de \textit{b\'usqueda local}.
\item Una heur\'istica \textit{GRASP}.
\end{enumerate}

El enfoque estará puesto en experimentar sobre cada uno de estos algoritmos, analizando su complejidad y la calidad de las solucion en el caso de las heur\'isticas.
Se intentará definir diferentes familias de grafos sobre los cuales poder obtener resultados concluyentes sobre el comportamiento del algoritmo.

\newpage
%-- Instrucciones --
\section{Instrucciones de uso}\label{sec:instrucciones}

Para compilar el proyecto completa: entrar en la carpeta src y correr make all.

Para compilar cada una de las partes por separado: en la carpeta src, estan las carpetas backtracking, grasp, greedy\_heuristic\_\{A,B,C\} y local\_search. Se puede compilar cada parte independiente, entrando en cada carpeta y ejecuenta make. 

Archivo MakeFile:

\begin{itemize}
  \item Hay un archivo Makefile.common que se usa como input para los makefile del resto de las carpetas.
  La sintaxis para incluirlo es: \texttt{include ../Makefile.common}
  
  \item Por defecto, compila el archivo que tenga el mismo nombre de la carpeta, y busca el .cpp y el .h.
	  Ejemplo, si estamos en la carpeta backtracking va a buscar backtracking.cpp y backtracking.h y lo compila en backtracking.
	  
  \item Siempre va a crear una carpeta OBJS en donde guarda los archivos .o
  
  \item Cuando se desean incluir archivos que se encuentren en la carpeta common, se debe anteponer COMMON\_OBJS al include:
  
  \begin{verbatim}
      COMMON\_OBJS := ClassName1 ClassName2
      include ../Makefile.common
  \end{verbatim}
  
  en donde ClassName es el nombre de la clase. Ejemplo, COMMON\_OBJS := Graph Edge

  \item Pueden agregar targets específicos que deseen luego del include ../Makefile.common, tal y como harían en un Makefile normal
  
\end{itemize}
\newpage

%-- Desarrollo --
\section{Desarrollo del TP}\label{sec:desarrollo}
  
  %-- Backtracking --
  \subsection{Backtracking}\label{subsec:backtracking}
  Debido a la dificultad computacional del problema, no existe a\'un una soluci\'on exacta de tiempo polinomial, y a pesar de que nos entretuvimos
discutiendo, nosotros no pudimos encontrarla tampoco. En su defecto implementamos un algoritmo de backtracking que recorre todos los caminos
posibles de $u$ a $v$ y se queda con el de menor $\omega_2$ tal que $\omega_1 \leq K$. 

El algoritmo funciona de la siguiente manera: en un momento dado va a tener construido un camino $P = [v_1 = u, \dots, v_{i-1}]$, y toma un nodo
$v_i$ aun no visitado de la adyacencia de $v_{i-1}$, lo marca como visitado y lo agrega a $P$. Si $\omega_1(P)$ se pasa de $K$, este camino ya no
nos sirve y no sigo avanzando en la recursión. Caso contrario, se llama a la recursión sobre el camino aumentado.

En el caso en que $v_i$ = $v$, sabemos que tenemos una solución candidata. Comparamos $\omega_2(P)$ con el menor $\omega_2$ entre las soluciones
candidatas, y si es mejor solución, se guarda. Cuando hemos llegado a $v$ no hace falta llamar a la recursión.

Siempre antes de retornar tras explorar un nodo de un camino, se lo marca como no visitado, para que pueda ser recorrido en otros caminos que se
recorran.

A continuaci\'on, escribimos el pseudoc\'odigo de la funci\'on \texttt{main}.
\begin{algorithm}[H]
\caption{$main$()}
\begin{algorithmic}[1]
  \State Camino mejorSolucion
  \State $\omega_2$(mejorSolucion) $\leftarrow$ $\infty$
  \State Camino ramaActual $\leftarrow$ []
  \For{i \textbf{en} 1 \textbf{a} n}
    \State visitados[i] = False
  \EndFor
  \State backtrack( u, null )
  \If{$\omega_2$(mejorSolucion) $<$ $\infty$}
    \State imprimir(mejorSolucion)
  \Else{}
    \State imprimir("no")
  \EndIf
\end{algorithmic}
\end{algorithm}

\begin{algorithm}[H]
\caption{$backtrack$(Nodo actual, Nodo padre)}
\begin{algorithmic}[1]
  \State ramaActual.push( actual )
  \State visitados[actual] $\leftarrow$ true
  \If{$\omega_1$(ramaActual) $< K$}
    \If{actual = v \textbf{and} $\omega_2$(ramaActual) $< \omega_2$(mejorSolucion)} 
      \State mejorSolucion $\leftarrow$ ramaActual
      \ElsIf{actual $\neq$ v \textbf{and} $\omega_1$(ramaActual) $\leq$ K}
      \For{\textbf{cada} Nodo n \textbf{en} adyacentes( actual ) }
	\If{\textbf{no} visitado[n] }
	  \State backtrack( n, actual )
	\EndIf
      \EndFor
    \EndIf
  \EndIf
  \State ramaActual.pop( actual )
  \State visitados[actual] $\leftarrow$ false
\end{algorithmic}
\end{algorithm}

Complejidad:

Analicemos cuantas llamadas se hacen a $backtrack$. Se exploran caminos de a los sumo $n$ nodos. El primer nodo está fijo en $u$. El segundo nodo
pertenece a la adyacencia del primero, que en peor caso tiene tamaño $n - 1$. El tercero pertenece a la adyacencia del segundo, que no hayan sido
visitados, cuyo tamaño es a lo sumo $n - 2$. Y asi sucesivamente. En peor caso se llama a $backtrack$ $n!$ veces.

Analicemos el cómputo que se realiza en cada llamada a $backtrack$. Preguntar si un nodo está visitado es acceder a un arreglo en forma constante.
El costo de una función de peso asociada a un camino se va acumulando a medida que se agregan nodos. El costo se guarda y se puede acceder en
forma constante. Lo queda por estudiar es la complejidad del ciclo $for$ que recorre la adyacencia del nodo visitado. Guardamos una lista de
adyacencia por lo que podemos recorrer la adyacencia de cualquier nodo con complejidad lineal en relación a su tamaño. El tamaño de la adyacencia
es a lo sumo $m$. Para cada nodo adyacente se efectúan operaciones constantes, a excepción de las llamadas a $backtrack$. El costo de estas
llamadas lo estamos calculando por separado.

En definitiva hacemos $O(n!)$ llamadas a una función que de por si toma $O(m)$ operaciones. La complejidad del algoritmo es $O(m*n!)$.

Experimentación:

Generamos en primera instancia grafos con una cantidad de nodos pequeña - entre 3 y 15. La cantidad de aristas, es decir, la densidad del grafo,
se eligió aleatoriamente entre 0 y (n * (n - 1)) / 2, la cantidad máxima posible. La distribución de las aristas, sus pesos, y el valor de $k$
también se eligieron de forma aleatoria.

A continuación presentamos los tiempos de ejecución de nuestro algoritmo frente a estas instancias. 

\includepdf[pages={1}]{imagenes/backtracking-aleatorio.pdf}

Para tamaños de grafo menores que 40, los tiempos de ejecución fueron despreciables. Se puede apreciar el carácter exponencial del tiempo de
ejecución de nuestro algoritmo en función del tamaño de entrada.

\fixme{Hacer mas experimentación? dejando fijo n y m? fue dificil sacar resultados para eso}

  \newpage

  %-- Greedy --
  \subsection{Greedy}\label{subsec:greedy}
  \subsubsection{Descripción}

Un algoritmo goloso usa una heur\'istica que consiste en elegir, en cada paso, una soluci\'on \'optima local entre un conjunto de opciones, esperando encontrar al final la soluci\'on \'optima global. En general estos algoritmos son eficientes y simples de dise\~nar e implementar, pero puede ser que nunca lleguen a la soluci\'on \'optima del problema. 

De acuerdo a la definici\'on de Brassard\footnote{\label{Brassard}Brassard G., Bratley P., {\it Fundamental of Algorithmics}, Prentice Hall, 1996. (c)}, un algoritmo goloso se compone de los siguientes elementos: 

\begin{enumerate}
 \item Un conjunto de candidatos que ya han sido considerados y seleccionados. 
 \item Un conjunto de candidatos considerado y rechazados. 
 \item Una funci\'on que comprueba si cierto conjunto de candidatos constituye una soluci\'on a nuestro problema, ignorando si es o no \'optima por el momento. 
 \item Una funci\'on de factibilidad, que me dice si es posible o no completar el conjunto a\~nadiendo otros candidatos para obtener al menos una soluci\'on de nuestro problema. 
 \item Una funci\'on selecci\'on que indica en cualquier momento cu\'al es el m\'as prometedor de los candidatos restantes, que no han sido seleccionados ni rechazados. 
 \item Una funci\'on objetivo, que da el valor de la soluci\'on que hemos hallado. 
\end{enumerate}

Lo que busca el algoritmo goloso es encontrar el conjunto de candidatos que constituya una soluci\'on, y que optimice el valor de la funci\'on objetivo. Este algoritmo avanza paso a paso. Inicialmente, el conjunto de elementos seleccionados est\'a vac\'io. Entonces, en cada paso se considera a\~nadir a este conjunto el mejor candidato sin coniderar los restantes, de acuerdo a nuestra funci\'on selecci\'on. Si el conjunto ampliado de candidato seleccionados ya no fuera factible, rechazamo el candidato que estamos considerando en ese momento. Sin embargo, si el conjunto aumentado sigue siendo factible, entonces a\~nadimos el candidato actual al conjunto de candidatos selecccionados, en donde pasar\'a a estar desde ahora en adelante. Cada vez que se ampl\'ia el conjunto de candidatos seleccionados, comprobamos si \'este constituye ahora una soluci\'on para nuestro problema. A partir de este esquema, al agregar siempre subsoluciones \'optimas a mi conjunto, al finalizar lo que se espera encontrar es la soluci\'
on \'optima. 

El algoritmo de Dijsktra para encontrar caminos m\'inimos en un grafo pesado es un algoritmo goloso, que funciona y es correcto, como lo fue demostrado por Bassard en el libro mencionado.

Dado un grafo $G=(V,X)$, Dijkstra guarda un conjunto $S$ de nodos que ya fueron recorridos y un vector $\pi$ con la distancia m\'inima de un nodo $u$ a todos los de $S$. En cada fase de Dijkstra, se selecciona un nuevo nodo de $V\backslash S$ cuyo valor en $\pi$ sea m\'inima y lo a\~nadimo a $S$, actualizando si es necesario $\pi$. Al finalizar, $\pi$ es el vector con la m\'inima distancia a todos los nodos. 

Entonces, nosotros para resolver el problema vamos implementar Dijkstra con tres funciones objetivo diferentes, que toman una arista y devuelven un peso para ella: 

\begin{enumerate}
  \item $f_A(e) = \omega_1(e)$
  \item $f_B(e) = \omega_2(e)$
  \item $f_C(e) = \omega_1(e)\omega_2(e)$
\end{enumerate}

Luego el pseudoc\'odigo de Dijkstra modificado con la nueva definici\'on de distancia queda dado por: 

\begin{algorithm}
In: Grafo $G = (V,X)$, nodo inicial $v_0$, ObjectiveFunction $f$ \newline
Out: Arreglo $\pi$ con camino m\'inimo en funci\'on de $f$ a cada nodo. 
\begin{algorithmic}[1]
\State $\pi(v) = \infty$ \quad $\forall v \in V$
\State $\pi(v_0) = 0$
\State $S = \emptyset$
\For{$i = 1 \dots n-1$}
    \State $v \leftarrow $ nodo de $V\backslash S$ de m\'inimo $\pi$. 
    \For{{\bf each} $w \in V\backslash S$ adyacente a $v$}
      \State $\pi(w) = \min( \pi(w), \pi(v) + f((v,w)))$
    \EndFor
    \State $S = S \cup \{v\}$
\EndFor
\State \textbf{retornar} $\pi$
\end{algorithmic}
\end{algorithm}

La modificaci\'on est\'a la l\'inea 7, que en vez de sumar a $\pi(v)$ el peso de la arista, como es en el algoritmo original, le suma el valor de una funci\'on que define el peso de la arista. Esto nos permite mucha flexibilidad a la hora de cambiar la ``decisi\'on golosa''.

\subsubsection{Complejidad}

Veamos que nuestro algoritmo es el mismo que el de Dijkstra salvo por que ejecuta la funci\'on $f((v,w))$, que al hacer operaciones que son constantes en el tiempo no altera la complejidad de final del algoritmo. 

De acuerdo al libro de Brassard antes mencionado, demuestran que la complejidad de Dijkstra implementada sobre un heap es $O(m \log n)$, y como esta es la mejor complejidad que se puede conseguir, nosotros lo implementamos sobre un \texttt{priority\_queue} de $C++$ que nos garantiza las mismas complejidades de un heap\footnote{http://www.cplusplus.com/reference/queue/priority\_queue/priority\_queue/}. 

%-- Goloso A --
\clearpage
\subsubsection{Familias Malas: Greedy A}\label{subsubsec:greedy-a}
Dado un grafo $G = (V,E)$, obtenemos el camino m\'inimo entre $u$ y $v$ seg\'un $\omega_1$. 

A continuación definimos una familia de grafos en los cuales nuestro algoritmo puede devolver resultados muy malos.
\ponerGrafico{imagenes/maloGreedyA.png}{}{0.5}{malo-para-greedy-a}

Para ir de 1 a 4 hay dos caminos posibles: ($C_1$) $1 \rightarrow 2 \rightarrow 4$; ($C_2$) $1 \rightarrow 3 \rightarrow 4$

\begin{eqnarray}
 \omega_1(C_1) &=& 2 	\\ 
 \omega_2(C_1) &=& 2x	\\
 \omega_1(C_2) &=& 2x	\\
 \omega_2(C_2) &=& 2
\end{eqnarray}

Supongamos que K vale 2x, es decir, los dos caminos son válidos. Nuestro algoritmo elige $C_1$.
$\frac{\omega_2(C_1)}{\omega_2(C_2)} = x$.
Como x lo podemos variar, este cociente puede ser tan grande como queramos. Es decir que el algoritmo goloso puede devolver una solución
arbitrariamente alejada de la óptima.

%-- Goloso B --
\clearpage
\subsubsection{Familias Malas: Greedy B}\label{subsubsec:greedy-b}
Dado un grafo $G = (V,E)$, obtenemos el camino m\'inimo entre $u$ y $v$ seg\'un $\omega_2$. 

A continuación definimos una familia de grafos en los cuales nuestro algoritmo puede devolver resultados muy malos.
\ponerGrafico{imagenes/maloGreedyB.png}{}{0.5}{malo-para-greedy-b}

Para ir de 1 a 4 hay dos caminos posibles: ($C_1$) $1 \rightarrow 2 \rightarrow 4$; ($C_2$) $1 \rightarrow 3 \rightarrow 4$

\begin{eqnarray}
 \omega_1(C_1) &=& 4 	\\ 
 \omega_2(C_1) &=& 2	\\
 \omega_1(C_2) &=& 2	\\
 \omega_2(C_2) &=& 2x
\end{eqnarray}

Supongamos que K vale 2. Nuestro algoritmo elige $C_1$, pero al no ser válido, se ve obligado a devolver ``no". Pero $C_2$ era una solución
válida. Ésto se cumple para cualquier valor de x.

\clearpage
%-- Goloso C --
\subsubsection{Familias Malas: Greedy C}\label{subsubsec:greedy-c}
Dado un grafo $G = (V,E)$, obtenemos el camino m\'inimo entre $u$ y $v$ seg\'un $\omega_1\omega_2$. 

Esta heurística se puede comportar de la misma forma que el Greedy B, si consideramos la familia mala de grafos desarrollada para el Greedy B,
restringiendonos a valores de x mayores a 2. Se eligiría $C_1$  para minimizar el producto de las funciones de peso. Como no es válido,
se deberá devolver ``no", a pesar de que $C_2$ era válido. 

\subsubsection{Experimentación}
Resulta importante destacar las decisiones que tomamos para la generación de grafos. Nos cercioramos primero de que exista un camino entre $u$ y
$v$ que cumpla con la restricción de $k$. Luego fuimos agregando caminos, como para dar varias opciones a los algoritmos, con la particularidad
de que eran $balanceados$. Ésto significa que si el peso según $\omega_1$ del camino era bajo, entonces le asignábamos un $\omega_2$ alto.
Más aún, para todo camino $C$ entre $u$ y $v$, $\omega_1(C)$ y $\omega_2(C)$ estaban distribuidos simetricamente con respecto a $k$.
Es decir, el camino $P$ más corto según $\omega_2$ que cumpla con la restricción de $k$ era tal que $\omega_1(P)$ = $\omega_2(P)$ = $k$.
En particular, nosotros insertábamos este camino $P$ en el grafo, de modo que llevamos cuenta del valor de la solución óptima. Ésto nos va a
facilitar evaluar la calidad de una solución cuando la instancia es muy grande como para correr el backtracking.


\begin{figure}[H]
\begin{center}
\includegraphics[angle=0, scale=.75]{imagenes/tiempos_greedy_magic/greedy_A_2014-06-27_02-37-46.pdf}
\label{grafico local}
\end{center}
\end{figure}

Se puede interpretar la curva de tiempo de ejecución vs. tamaño de entrada como una función cercana a la lineal, aunque se puede percibir
que está ligeramente por encima de ésta.

\begin{figure}[H]
\begin{center}
\includegraphics[angle=0, scale=.75]{imagenes/calidad_greedy_2014-06-27_09-50-48.pdf}
\label{grafico local}
\end{center}
\end{figure}

Lo que salta a simple vista es el bajo peso según $\omega_2$ de la heurística B. ¡Pareciera ser mejor solución que la óptima! Sin embargo,
recordando lo expuesto previamente, nos damos cuenta que si el peso según $\omega_2$ es menor que el de la óptima, entonces el peso según
$\omega_1$ es mayor que el de la óptima, es decir, que $k$. Por lo tanto no son soluciones válidas.
Ésto era de esperar - la heurística B sólo va construyendo su solución sin reparar siquiera en $\omega_1$.

Por otro lado tenemos la heurística A, que sólo mira $\omega_1$ y por lo tanto termina bastante alejada de la solución óptima. La heurística
C resultó ser un acierto, se acerca bastante a la solución válida, pero no descuida el mantenerse dentro de la cota de $k$.

  \newpage

  %-- Local Search --
  \subsection{Local Search}\label{subsec:local}
  % Búsqueda local
La búsqueda local es un método heurístico para mejorar una dada solución factible a un problema. Se basa en, a partir de  una solución inicial $s$, mejorar esa solución iterativamente, tomando la mejor solución de un conjunto de vecinos de $s$. El conjunto de los vecinos se determina mediante algún criterio y se usa una función objetivo para comparar las soluciones en la vecindad de $s$ y discernir cual es la mejor.

La búsqueda local se puede expresar de la siguiente manera:\\\\
\hspace*{1 cm} Sea $s \in S$ una solución inicial\\
\hspace*{1 cm} Mientras exista $s' \in N(s)$ con $f(s') < f(s)$:\\
\hspace*{2 cm} $s \leftarrow s'$

Siendo $N(s)$ la vecindad de $s$ y $f$ la función objetivo que se quiere minimizar. Una versión de la búsqueda local se encarga además de que el vecino que elige minimice la función objetivo entre todas los vecinos.

% Solución inicial
\subsubsection{Solución inicial}

Como solución inicial se utilizó la heurística \texttt{Greedy} desarrollada en el punto anterior, consistente en elegir la mejor de tres corridas del algoritmo de Dijkstra con distintas funciones de peso.

% Definición de la vecindad
\subsubsection{Definición de la vecindad}

En cada iteración definimos la vecindad de $s$, $N(s)$, de la siguiente manera: dado el grafo G, y una solución formada por un camino $c \in G$, definimos sus soluciones vecinas como aquellas resultantes de tomar un subcamino $d_{n_1,n_2} \subseteq c$ entre un par de nodos $n_1,n_2$ cualesquiera, y reemplazarlo por otro camino $d_{n_1,n_2}^*$ en $G$, de tal forma que el camino $c^* = c - d_{n_1,n_2} + d_{n_1,n_2}^*$ resultante cumpla: 

\begin{itemize}
\item $\omega_1(c^*)$ $<=$ $K$
\item $\omega_2(c^*)$ $<$ $\omega_2(c)$
\item $c^*$ no contiene ciclos
\end{itemize}

De la forma descripta, dada una solución formada por un camino $c$ definimos su vecindad como el conjunto $S^*$ de todos los caminos $c^*$ posibles.

Corremos previamente $n$ veces Dijkstra, una vez desde cada nodo. Guardamos los resultados en una matriz tal que en la posición $[i][j]$ guarda el camino mínimo del nodo $i$ al $j$. Cada camino $d_{n_1,n_2}^*$ es el camino mínimo entre $n_1$ y $n_2$ según $\omega_2$, se obtiene de la matriz. Debe además cumplir $\omega_1(c^*) \leq K$. De todos los $c^*$, se elige el que minimice $\omega_2$. Esto se enmarca en la elección de vecinos utilizando el método \textit{Steepest Descent} consistente en elegir al vecino que minimice la función objetivo.

Puede darse el caso de que, al reemplazar un camino $d_{n_1,n_2}$ por $d_{n_1,n_2}^*$, existan nodos en este último que también existan en $c^* = c - d_{n_1,n_2}$, de tal forma que el camino $c^*$ resultante contenga ciclos. Para resolver este inconveniente, luego de efectuar cada reemplazo, se llama al método removeCycles, el cual toma un camino, ya sea con ciclos o sin ciclos, y se encarga de construir una versión del mismo la cual no contenga ciclos.

Ahora, para calcular $\omega_2(c^*)$, lo calculamos como

\[
\omega_2(c) - \omega_2(d_{n_1,n_2}) + \omega_2(d_{n_1,n_2}^*)
\]

, valores que ya tenemos calculados.

\subsubsection{Pseudocódigo}

El algoritmo está implementado en la función \texttt{main}:

\begin{algorithm}[H]
\caption{$main$(int tipo\_solucionInicial, Graph g, Nodo n1, Nodo n2)}
\begin{algorithmic}[1]
  \State crearMatrizCaminosMinimos(g)
  \State Solution solucion = \texttt{Greedy(g, n1, n2)}
  
  \If{$\omega_1$(solucion) $\leq$ K}    
    \While{True}    	        
    	\State Solution nuevaSolucion = dameMejorVecino(solucion)
	\If{nuevaSolucion == NULL} 
		\State break	
	\EndIf    
	\State solucion = nuevaSolucion	
    \EndWhile
  \EndIf
\end{algorithmic}
\end{algorithm}

\begin{algorithm}[H]
\caption{$dameMejorVecino$(Solution solucionOriginal)}
\begin{algorithmic}[1]	
          \State Solution mejorSolucion = \&solucionOriginal
	  \State vector<int> nodos = \&nodos(solucionOriginal)
	  \For{i=0; i $<$ size(nodos); i++}
	  	\For{j=i+1; j $<$ size(nodos); j++}
			\State Solution subSolucion = crearSubSolucionEntre(solucionOriginal, nodos[i], nodos[j])
			\State Solution solucion\_ij = dameCaminoResueltoEntre(nodos[i], nodos[j])
                        \State removerCiclos(solucion\_ij)
			\State Solution nuevaSolucion\_$\omega_2$ = $\omega_2$(solucionOriginal) - $\omega_2$(subSolucion) + $\omega_2$(solucion\_ij)
			\State Solution nuevaSolucion\_$\omega_1$ = $\omega_1$(solucionOriginal) - $\omega_1$(subSolucion) + $\omega_1$(solucion\_ij)			
			\If{nuevaSolucion\_$\omega_2$ $<$ $\omega_2$(mejorSolucion) \&\& nuevaSolucion\_$\omega_1$ $\leq$ K)}
				\State mejorSolucion = crearSolucionReemplazandoCamino(solucionOriginal, solucion\_ij)
			\EndIf
		\EndFor
	\EndFor
	\State return mejorSolucion
\end{algorithmic}
\end{algorithm}

\begin{algorithm}[H]
\caption{$crearSolucionReemplazandoCamino$(Solution orig, Solution sub)}
\begin{algorithmic}[1]	
	 \State Solution res
	 \State res = obtenerCaminoHasta(nodos(sub)[0])
	 \State res += sub
	 \State int subSize = size(nodos(sub))
	 \State res += obtenerCaminoDesde(nodos(sub)[subSize-1])	  
	\State return res
\end{algorithmic}
\end{algorithm}

\begin{algorithm}[H]
\caption{$removerCiclos$(Solution solution)}
\begin{algorithmic}[1]	
	 \State int nextNodes[nodeCount]
	 \For{i=0; i $<$ nodeCount; i++}
	 	\State nextNodes[i] = 0
	\EndFor
	 \For{i=0; i $<$ solution.path.size(); i++}
		 \State edge = solution.path[i]
	 	\State nextNodes[edge.fromNode] = edge.toNode
	\EndFor	
	 \State vector$<int>$ newPath
	 \State firstNode = solution.path[0].fromNode
	 \State lastNode = solution.path[solution.path-1].toNode
	 \State next = firstNode
	 \State newPath.push(firstNode)
	 \While{nextNodes[next] != lastNode}
	 	\State next = nextNodes[next]
		\State newPath.push(next)
	\EndWhile
	\State newPath.push(lastNode)
	\State Solution newSolution
	\For{i=0; i $<$ newPath.size()-1; i++}	
		\State edge = solution.getEdgeBetween(newPath[i], newPath[i+1]
		\State newEdge = Edge(edge)
		\State newSolution.addEdge(edge)
	\EndFor
	\State return newSolution
\end{algorithmic}
\end{algorithm}

% Complejidad
\subsubsection{Complejidad}

Implementamos el algoritmo en la función $main$. La complejidad del algoritmo resulta la suma de obtener la solución inicial y el ciclo que se usa para mejorarla buscando un mejor vecino en cada iteración. 
Además, en $main$, inicialmente, se llama a una función crearMatrizCaminosMinimos\footnote{\label{$crearMatrixCaminosMinimos$}: Estamos haciendo uso de esta función que no se encuentra en el código fuente como tal, pero es para facilitar la legibilidad del pseudocódigo. En el código fuente se crea la matriz de camino mínimos cuando se inicializa NeighbourhoodSelector en la función main en localsearch.cpp.}, que abordaremos más adelante.

Creamos la solucion inicial usando Greedy. Esta función como explicamos en el apartado anterior, devuelve un camino mínimo entre 2 nodos usando Greedy A, B y C. Cada una de las 3 funciones Greedy usan la función resolverConDijkstra que ejecuta Dijkstra para encontrar todos los caminos mínimos entre n1 y los demás nodos y después hace un recorrido inverso desde n2 hasta n1 para dar con el camino mínimo entre ellos. Como se comprobó en el apartado anterior, la complejidad de Dijkstra es $O(m log(n))$ y el recorrido inverso es $O(n)$, por lo que en total la complejidad de obtenerSolucionInicial es $O(m log(n))$. Cabe notar que la función objetivo usada en Dijkstra no influye en la complejidad, ya que solo compara los valores de $\omega_1$ y $\omega_2$, por lo que tiene complejidad $O(1)$.

Para mejorar la solución usamos un ciclo y en cada iteración obtenemos el mejor vecino del camino actual usando dameMejorVecino. Ejecutamos el ciclo mientras en la última iteración hayamos encontrado una mejora al camino actual. 
Dado que un vecino consiste en reemplazar la porción del camino actual que une 2 nodos $n_1$ y $n_2$ por el camino mínimo según $\omega_2$ entre ellos, y que cada vez que se hace el reemplazo se disminuye $\omega_2$ del camino actual, se pueden hacer a lo sumo tantos reemplazos como caminos mínimos entre todo par de nodos del grafo existan. 
La cantidad de caminos mínimos entre todo par de nodos de un grafo es $n \times (n-1) / 2$, ya que cada nodo tiene un camino mínimo hacia todos los demás y no a sí mismo. Esto es, si hay n nodos, el nodo $n_1$, tiene un camino mínimo hacia los nodos $n_2$, ..., $n_n$, el nodo $n_2$ tiene un camino hacia $n_3$, ..., $n_n$ ya que no se vuelve a contar el camino entre $n_1$ y $n_2$ y así hasta $n_{n-1}$ que tiene un camino hasta $n_n$.
Luego, la cantidad máxima de iteraciones es $n \times (n-1) / 2$.

Para analizar la complejidad de cada iteración hay que analizar la complejidad de dameMejorVecino. 
Lo primero que hace la función es obtener el arreglo de nodos de la solución pasada por parámetro, que llamamos solucionOriginal. Cómo el camino de solucionOriginal está representado como una lista de ejes, lo que hace es iterar por todos los ejes y tomar el nodo1 del eje y al final adicionar el nodo2 del último eje. Por ende, tomando $t$ como la cantidad de nodos en el camino, en cada iteración, por cada eje del camino, se toma un nodo y esto se hace $t$-1 veces y luego se añade el nodo final. Como $t$ puede ser a lo sumo $n$, entonces la complejidad de esto resulta $O(n^2)$.
Luego de ejecuta un doble $for$ de $t \times (t-1) / 2$ iteraciones, que en cada iteración intenta mejorar el mejor camino encontrado hasta el momento, que llamamos mejorSolucion. Inicialmente mejorSolucion es igual a solucionOriginal. 
Para encontrar una mejor solución en cada iteración hacemos lo siguiente:

\begin{itemize}
\item Creamos el subcamino de solucionOriginal entre los nodos $n_1$ y $n_2$.
\item Obtenemos el camino mínimo entre los nodos $n_1$ y $n_2$.
\item Obtenemos un nuevo camino reemplazando el el subcamino entre los nodos $n_1$ y $n_2$ por el camino mínimo entre ellos.
\end{itemize}

Para crear el subcamino de solucionOriginal entre $n_1$ y $n_2$ usamos crearSubSolucionEntre. Esta función recibe un par de nodos y un camino y devuelve el subcamino que une a los nodos. Para esto tiene que recorrer a lo sumo todos los nodos del camino, o sea $t$, que puede ser a lo sumo $n$.

Para obtener el camino mínimo entre los nodos $n_1$ y $n_2$ usamos una optimización que es que en la función $main$, al comienzo, ejecutamos crearMatrizCaminosMinimos que crea una matriz de caminos mínimos entre todo par de nodos del grafo. Generamos esta matriz ejecutando Dijkstra para todos los nodos usando como función objetivo $\omega_2$. Usamos esta matriz para obtener en $O(1)$ el camino mínimo entre los nodos $n_1$ y $n_2$.
Como crearMatrizCaminosMinimos ejecuta un Dijkstra por cada nodo, su complejidad es $O(n \times (m log n))$.

Para obtener el nuevo camino reemplazando el subcamino entre $n_1$ y $n_2$ por el camino mínimo entre ellos, usamos crearSolucionReemplazandoCamino.
Lo que hacemos es generar una nueva solución que es la concatenación de 3 caminos: el camino mínimo entre $n_1$ y $n_2$ y los dos pedazos del camino original sin el subcamino que unía $n_1$ y $n_2$. Para crear el camino, hay que recorrer el solucionOriginal y añadir todos los nodos hasta $n_1$ y luego añadir todos los nodos del camino mínimo y al final añadir todos los nodos desde $n_2$ hasta el final de solucionOriginal. Por ende es como iterar sobre el nuevo camino que a lo sumo puede tener $n$ nodos y por ende la complejidad resulta $O(n)$.

Como es posible que la solución creada a partir de unir los caminos contenga ciclos, usamos la función removerCiclos que se encarga de remover cualquier ciclo en la nueva solución. Esta función itera sobre los nodos de la solución y almacena en un arreglo de enteros el nodo siguiente a cada nodo. De modo que cada elemento i corresponde al nodo i y su valor es el nodo siguiente a ese nodo i. En el caso de que exista un ciclo y un nodo i esta al comienzo de un ciclo, primero el nodo i tendrá como siguiente a otro nodo j, pero cuando se itera sobre la solución y se llega nuevamente al nodo i, al asignarle su siguiente, ese nodo ya no será j, sino otro nodo mas adelente en la solución. Con el arreglo creado de esa manera podemos reconstruir la solución sin ciclos. El camino se obtiene recorriendo el arreglo como una lista enlazada, donde cada elemento tiene el índice del siguiente. Usando este arreglo creamos una nueva solución, que es un vector de ejes con $omega_1$ y $omega_2$ asignados. Los ejes los construimos usando los nodos del arreglo y obtenemos los valores de $omega_1$ y $omega_2$ de la solución recibida por parametro. La complejidad de remover los ciclos resulta igual a la suma de construir el arreglo, que es $O(n)$, ya que a lo sumo puede tener n vertices y luego crear la nueva solución que es $O(n^2)$, ya que hay que crear hasta $n-1$ ejes y por cada eje hay que obtener sus valores de $omega_1$ y $omega_2$ de la solución original. Luego en total es $O(n^2)$. 

Entonces, resulta ser que encontrar el mejor vecino, consiste en ejecutar el doble $for$ de algo que tiene complejidad $O(n + 1 + n + n^2)$, o sea $O(n^2)$, entonces en total es $O(n^2 \times n^2) = O(n^4)$.

Como habíamos explicado que dameMejorVecino se ejecuta en un ciclo de hasta $n \times (n-1) / 2$ iteraciones, o sea $O(n^2)$, y entonces resulta que la complejidad de encontrar el mejor vecino es $O(n^2 \times n^4) = O(n^6)$. 

Finalmente la complejidad total resulta ser la suma de las complejidades de crearMatrizCaminosMinimos, 2 veces obtenerSolucionInicial y $n^2$ veces dameMejorVecino. Es decir, $O(n \times (m log n) + 2 \times (m log n) + n^6) = O(n^6)$.

% Familias malas
\subsubsection{Familias malas}

Veamos que nuestra heurística de búsqueda local puede quedar arbitrariamente lejos de la solución óptima.

Presentamos la siguiente familia de grafos:

\ponerGrafico{imagenes/maloLocalA.png}{}{0.5}{malo-para-local-a}

Para ir de 1 a 5 hay tres caminos posibles: ($C_1$) $1 \rightarrow 2 \rightarrow 5$; ($C_2$) $1 \rightarrow 3 \rightarrow 5$;
($C_3$) $1 \rightarrow 4 \rightarrow 5$;

\begin{eqnarray}
 \omega_1(C_1) &=& 4	\\ 
 \omega_2(C_1) &=& 2	\\
 \omega_1(C_2) &=& 2	\\
 \omega_2(C_2) &=& 2x   \\
 \omega_1(C_3) &=& 6	\\
 \omega_2(C_3) &=& 1
\end{eqnarray}

Supongamos que K vale 4.
Como decíamos, partimos del camino mínimo entre $u$ y $v$ de acuerdo a $\omega_1$, $C_2$. El algoritmo va a intentar intercambiar $C_2$ por
$C_3$, el camino que minimiza $\omega_2$. Sin embargo, como éste se pasa del límite K, el algoritmo no puede seguir y devuelve $C_2$. Sin
embargo $C_1$ era una solución mejor.

$\frac{\omega_2(C_2)}{\omega_2(C_1)} = x$.

Haciendo crecer el valor de $x$ podemos encontrar grafos en los que nuestro algoritmo devuelve una
solución arbitrariamente lejos de la óptima.

En general, la heurística falla cuando los caminos mínimos por $\omega_2$ entre todo par de nodos tienen un valor de $\omega_1$ demasiado alto, que deja a la solución sin vecinos factibles ya que éstos se sobrepasan de la restricción de $K$.
\newpage

\subsubsection{Experimentación}

Pudimos analizar el tiempo de ejecución corriendo el algoritmo en grafos de un tamaño considerable. Utilizamos instancias del tipo \textit{mágico}, variando el valor de $n$ y eligiendo $m$ al azar entre todos los posibles valores dado $n$.  A continuación presentamos un gráfico.

\begin{figure}[H]
\begin{center}
\includegraphics[angle=0, scale=.75]{imagenes/local_search_time.pdf}
\label{grafico local}
\end{center}
\end{figure}


Se puede percibir el carácter no lineal de la complejidad. Sin embargo no se evidencia a luces claras la cota teórica de $n^6$ desarrollada
previamente. Puede ser necesario un tamaño mayor para poder apreciar esa complejidad, o quizás la cota teórica no se alcanza en la práctica.
Probablemente haya un factor que se amortice, habiendo escapado de nuestro escrutinio.

La conclusión que se obtuvo tras haber terminado todas las experimentaciones es que \texttt{local\_searc} hace generalmente pocas iteraciones, por lo que el tiempo de ejecución se asemeja al de \texttt{Greedy}, cuya complejidad teórica es $O(m*log(n)$.

\newpage
Luego se propuso analizar la calidad de la solución encontrada.
Experimentamos con distintas densidades de grafos. Primero con grafos densos, es decir con una cantidad de aristas del orden de $n^2$. Luego con
grafos de densidad media, con una cantidad de aristas del orden de $\sqrt{n} \times n$. Finalmente probamos con grafos ralos, con una cantidad de aristas
del orden de la cantidad de nodos. Para cada grupo de grafos, analizamos diferentes rangos de tamaños de entrada, con el objetivo de demostrar
que el algoritmo mantiene ciertas propiedades, más allá del tamaño.

Se generan cien instancias para cada tamaño de entrada. En los siguientes gráficos se pueden observar comparados el mejor resultado, el peor
resultado, y el resultado promedio para cada tamaño de entrada. Al ser instancias del tipo \textit{mágico}, podemos saber el valor de la solución óptima - $n-1$ - y graficarla.

\begin{figure}[H]
\begin{center}
\includegraphics[angle=0, scale=.70]{imagenes/calidad_local_search_2014-06-27_16-02-57.pdf}
\label{grafico local}
\end{center}
\end{figure}

\begin{figure}[H]
\begin{center}
\includegraphics[angle=0, scale=.70]{imagenes/calidad_local_search_2014-06-27_08-58-52.pdf}
\label{grafico local}
\end{center}
\end{figure}

\begin{figure}[H]
\begin{center}
\includegraphics[angle=0, scale=.75]{imagenes/calidad_local_search_2014-06-27_08-54-45.pdf}
\label{grafico local}
\end{center}
\end{figure}

En los tres gráficos se puede notar, a medida que aumenta el tamaño de la entrada, una mayor amplitud de resultados. Ésto tiene cierta intuición. 
Se generan muchas instancias para cada tamaño. A medida que aumenta la cantidad de aristas y vértices, aumenta la variedad entre las instancias
generadas. Esto permite ver una gran diferencia de resultado entre instancias del mismo tamaño.
La solución media, no obstante, parece preservar cierta distancia relativa, cierta escala con respecto a la solución óptima.

La mejor solución encontrada siempre está muy cerca de la solución óptima. Ésto se evidencia particularmente en el tercer gráfico, lo que
atribuímos a la mayor densidad que le permite al algoritmo tener una vecindad amplia para recorrer. Sin embargo pareciera que la densidad también aumenta la amplitud y por lo tanto empeora la solución máxima. Puede tener sentido si se considera que al aumentar la cantidad de aristas que se deben generar cada una con su par de nodos y pesos, la diferencia entre dos instancias es combinatoria teniendo en cuenta todas las elecciones posbibles para cada arista.

A continuación se procede a estudiar la cantidad de iteraciones efectuadas por la búsqueda local, en relación a la cantidad de nodos del grafo.

\begin{figure}[H]
\begin{center}
\includegraphics[angle=0, scale=.70]{imagenes/local-iteraciones.pdf}
\label{grafico local}
\caption{Cantidad de iteraciones en instancias \textit{mágicas}, de densidad media}
\end{center}
\end{figure}

La cantidad de iteraciones realizadas presenta una gran amplitud, aún para tamaños de grafo similares. Tiene sentido si se considera que diferentes adyacencias de nodos, aún cuando la cantidad de nodos y aristas es la misma, puede permitir o no a la búsqueda local elegir una mejor solución vecina. La cantidad de iteraciones muestra una tendencia lineal a crecer. Sin embargo se considera que es un resultado más bajo de lo esperado, ya  parte importante de una heurística de búsqueda local es que la solución tenga movilidad suficiente y no quedarse estático cerca de la solución inicial. Se notó una gran cantidad de casos en los que la búsqueda local no pudo mejorar la solución inicial.

Nos propusimos mejorar la perspectiva de la búsqueda local implementada, analizando la cantidad de unidades de $\omega_2$ que se disminuyen en cada iteración de la búsqueda local. 

\begin{figure}[H]
\begin{center}
\includegraphics[angle=0, scale=.70]{imagenes/local_search-mejora.pdf}
\label{grafico local}
\caption{Mejora en cada iteración en instancias \textit{mágicas} de densidad media}
\end{center}
\end{figure}

Los resultados fueron sorprendentes. Como quedo expuesto anteriormente, una corrida de búsqueda local no suele hacer muchas iteraciones, pero cuando las hace, disminuye notablemente el valor de $\omega_2$ de la solución. Debe tenerse en cuenta que, siendo éstas instancias de tipo \textit{mágico}, el valor de $\omega_2$ de la solución óptima es $n-1$. De modo que lo que se ve en el gráfico es que el promedio de mejora en una iteración es del 60\% del valor de la solución óptima, proporción que se mantiene en las magnitudes de grafos consideradas.


  \newpage

  \subsection{GRASP}\label{subsec:grasp}
  %Grasp
La metaheurística Grasp consiste en generar varias soluciones iniciales a partir de una heurística golosa aleatorizada y correr búsqueda local sobre ellas. La solución final es la mejor encontrada tras todas las búsquedas locales.
Sea $S$ el conjunto de soluciones iniciales, el algoritmo que se usa es el siguiente:\\\\
\hspace*{1 cm} Mientras no se alcance el criterio de terminación:\\
\hspace*{2 cm} Obtener $s \in S$ mediante una heurística golosa aleatorizada.\\
\hspace*{2 cm} Mejorar $s$ mediante búsqueda local.\\
\hspace*{2 cm} Recordar la mejor solución obtenida hasta el momento.\\
%Solucion inicial
\subsubsection{Solución inicial}

Usamos Dijkstra como nuestra heurística golosa, pero modificado para agregarle aleatoriedad. El factor aleatorio consiste en que en cada iteración de Dijkstra, en vez de tomar el nodo no visitado que minimiza la función objetivo, tomamos uno de entre los $beta$\footnote{\label{$beta$}: Luego de experimentar con distintos valores, encontramos que $beta$=10 era un valor que presentaba suficiente aleatoriedad.} menores, al azar.

Dada la aleatoriedad de la solución inicial, es posible que ésta no sea factible. En caso de obtener una solución de ese tipo, no ejecutamos búsqueda local.

Utilizamos Dijkstra con la función de peso $\omega_1$. De esta forma se intenta dar variedad a las soluciones iniciales, pero al mismo tiempo que éstas sean factibles - es decir, que cumplan con la restricción de $\omega_1 < K$ - en su mayoría.

\subsubsection{Criterio de terminación}

Usamos 3 criterios de terminación distintos al mismo tiempo. De alcanzarse alguno de los criterios, se termina la ejecución del algoritmo.

Los criterios que usamos son:
\begin{itemize}
\item Cantidad máxima de iteraciones.
\item Cantidad máxima de iteraciones sin haber encontrado mejoras.
\item Cantidad máxima de iteraciones sin haber encontrado una solución inicial factible.
\end{itemize}

Parametrizamos estos criterios usando $n$. Para elegir valores adecuados realizamos distintas experimentaciones sobre grafos de tipo \textit{mágico} con baja, media y alta densidad. Los resultados obtenidos se pueden ver en los siguientes gráficos:

\begin{figure}[H]
\begin{center}
\includegraphics[angle=0, scale=.5]{imagenes/iteraciones-GRASP-A.png}
\label{Resultados experimentales A}
\end{center}
\end{figure}

\begin{figure}[H]
\begin{center}
\includegraphics[angle=0, scale=.5]{imagenes/iteraciones-GRASP-B.png}
\label{Resultados experimentales A}
\end{center}
\end{figure}

\begin{figure}[H]
\begin{center}
\includegraphics[angle=0, scale=.5]{imagenes/iteraciones-GRASP-C.png}
\label{Resultados experimentales C}
\end{center}
\end{figure}

Las experimentaciones A, B y C corren GRASP usando grafos de 200 nodos y 400, 4000 y 10000 ejes respectivamente. Cada una ejecuta 40000 iteraciones.

Usamos escala logarítmica para facilitar la visualización de los resultados.

Cada figura muestra como evoluciona el valor de $\omega_2$ a medida que aumentan las iteraciones. Además graficamos el camino óptimo con $\omega_2$ igual a la cantidad de nodos del grafo, o sea 200, para poder comparar con los valores obtenidos.

Nuestro razonamiento para tomar valores óptimos para los criterios fue elegirlos de manera que se adecúen a grafos de variada densidad. Analizando los gráficos obtuvimos que el algoritmo puede tardar hasta aproximadamente 10000 iteraciones sin encontrar mejoras. Ese valor lo obtuvimos del anteúltimo salto de la experimentación C y corresponde a $n^2 / 4$.

La máxima cantidad de iteraciones necesarias para encontrar un valor cercano al óptimo fue máxima en la experimentación C también. De alrededor de 20000 iteraciones, o $n^2 / 2$.

Para elegir un valor para la cantidad máxima de iteraciones sin haber encontrado una solución inicial factible, tomamos el mismo valor que para la cantidad máxima de iteraciones sin haber encontrado mejoras, o sea $n^2 / 4$. Nuestro razonamiento es que si no se encontró una solución inicial factible, tampoco se mejoró la última solución encontrada y por ende nos sirve para acotar el valor para ese criterio.

Además de encontrar valores adecuados para esos criterios, encontramos interesante como influye la densidad de los grafos en el comportamiento del algoritmo. A baja densidad le cuesta más encontrar una buena solución inicial y aun más mejorar la última solución encontrada. Razonamos que en la experimentación A hay pocos caminos, dada la baja densidad, y hay menos chances de encontrar los mejores caminos, ya que son pocos. Y creemos que por eso sucede que no se llega al óptimo luego de 40000 iteraciones, ya que posiblemente no exista ningún camino óptimo o las chances de encontrarlo son muy bajas.

El experimento que muestra el mejor comportamiento del algoritmo es el B, o sea usando densidad media. Logra encontrar el camino óptimo mucho antes de llegar a nuestra cota para la cantidad de iteraciones: tarda alrededor de 1000 iteraciones, muy por debajo de las 20000($n^2 / 2$) iteraciones que usamos como cota. Además logra encontrar un camino inicial bastante mejor que las otras dos experimentaciones($\omega_2$ = 250 frente a $\omega_2$=300+ de A y C). 

Por último, creemos que la experimentación C muestra un caso característico de GRASP, ya que se que mejora la solución escalonadamente hasta dar con la óptima luego de varias mejoras parciales.


\subsubsection{Búsqueda local}

La búsqueda local que realizamos es la misma que en el apartado anterior.

\subsubsection{Pseudocódigo}

El algoritmo está implementado en la función \texttt{main}:

\begin{algorithm}[H]
\caption{$main$(int tipo\_solucionInicial, Graph g, Nodo n1, Nodo n2)}
\begin{algorithmic}[1]
  \State crearMatrizCaminosMinimos(g)
  
  \State int $n = |nodes(g)|$
  \State int iteracionesSinMejorarCount = 0
  \State int iteracionesSinMejorarMax = n
  \State int iteracionesMax = $n * log(n)$
  \State int iteracionesSinInitialPathCount = 0
  \State int iteracionesSinInitialPathMax = n
  \State mejorSolucion = NULL
  \For{i=0; i$<$iteracionesMax; i++}
  	\State Solution solucion = obtenerSolucionInicial(tipo\_solucionInicial, g, n1, n2)	
	\If{$\omega_1$(solucion) $>$ K}
    		\State iteracionesSinInitialPathCount++
		\If{iteracionesSinInitialPathCount $\geq$ iteracionesSinInitialPathMax}
			break
		\EndIf
        \EndIf

  
	\If{$\omega_1$(solucion) $\leq$ K}    
	    \While{True}    	        
	    	\State Solution nuevaSolucion = dameMejorVecino(solucion)
		\If{nuevaSolucion == NULL} 
			\State break	
		\EndIf    
		\State solucion = nuevaSolucion	
	    \EndWhile
	  
	  \If{mejorSolucion == NULL}
	  	\State mejorSolucion = solucion
	  \ElsIf{$\omega_2$(solucion) $<$ $\omega_2$(mejorSolucion)}
	  	\State mejorSolucion = solucion
	  \Else
	  	\State iteracionSinMejorarCount++
	  \EndIf
	\EndIf
	
	\If{iteracionesSinMejorarCount $>$ iteracionesSinMejorarMax}
                \State break
        \EndIf
	  
    \EndFor
\end{algorithmic}
\end{algorithm}

\begin{algorithm}[H]
\caption{$obtenerSolucionInicial$(int tipo, Graph g, Nodo n1, Nodo n2)}
\begin{algorithmic}[1]
  \If{tipo == Greedy\_C}
	\State return resolverConDijkstraAleatorio(g, n1, n2, ObjectiveFunctionC)
  \EndIf
  \State return resolverConDijkstraAleatorio(g, n1, n2, ObjectiveFunctionA)
\end{algorithmic}
\end{algorithm}

Las demás funciones tienen el mismo pseudocódigo que en Búsqueda local.

\subsubsection{Complejidad}

El algoritmo se ejecuta en un ciclo hasta cumplir con alguno de los criterios de terminación. Dado que el criterio de mayor valor es la cantidad de iteraciones totales($iteracionesMax$), tomamos ese valor como cota para calcular la complejidad.
Cada iteración del ciclo es casi idéntica a la ejecución de búsqueda local. La única diferencia es cómo se obtiene la solución inicial. Para encontrar la solución inicial se usa resolverConDijkstraAleatorio. Esta función, en vez de tomar el nodo no visitado con $\omega_2$ mínimo, toma uno entre los $beta$ menores. Pero resulta que la complejidad no se altera con este cambio, ya que Dijkstra itera sobre todos los nodos de cualquier manera y lo único que cambia es el orden en que se toma el nodo no visitado. 

Vale hacer una aclaración que es que como se usa una cola con prioridad para los nodos no visitados en Dijkstra, para sacar uno entre los $beta$ menores, hay que remover los primeros $beta$ nodos de la cola y luego volver a agregar todos menos uno que es con el que nos quedamos. Como remover y agregar de la cola con prioridad toma $O(log(n))$, entonces para remover y agregar los $beta$ nodos se toma $O((beta + beta-1) * log(n))$, pero como $beta$ es constante entonces la complejidad resulta $O(log(n))$. 

Por lo tanto la complejidad de resolverConDijkstraAleatorio no resulta diferente resolverConDijkstra, usada en busqueda local y por ende la complejidad de cada iteración resulta igual a la complejidad de una ejecución de búsqueda local, o sea tiene complejidad $O(n^5)$.

Luego la complejidad total es $O(n^5 * iteracionesMax)$ = $O(n^5 * n*log(n))$ = $O(n^6 * log(n))$.

\subsubsection{Experimentación}

A continuación presentamos los resultados de la experimentación del tiempo de ejecución de la metaheurística GRASP.

\begin{figure}[H]
\begin{center}
\includegraphics[angle=0, scale=.75]{imagenes/grasp_2014-06-27_19-18-59.pdf}
\label{grafico local}
\end{center}
\end{figure}

No se pareciera estar frente a una polinomio de grado 6, sin embargo, es evidente que tiene mucha más concavidad que la gráfica de la búsqueda
local. Ésto proviene trivialmente del hecho de repetir una cantidad que puede llegar a ser n * log(n) veces la búsqueda local.

Pasamos a experimentar con la calidad de la solución.

\begin{figure}[H]
\begin{center}
\includegraphics[angle=0, scale=.75]{imagenes/calidad_grasp_2014-06-27_08-58-53.pdf}
\label{grafico local}
\end{center}
\end{figure}

\begin{figure}[H]
\begin{center}
\includegraphics[angle=0, scale=.75]{imagenes/calidad_grasp_2014-06-27_08-54-46.pdf}
\label{grafico local}
\end{center}
\end{figure}

Grata fue la sorpresa al ver a GRASP en acción. Al igual que la búsqueda local, el resultado mínimo de nuestro algoritmo está muy cerca del
óptimo. Pero esta vez hemos logrado reducir la amplitud entre nuestros resultados y de esta forma acercar todo el cuerpo de nuestras soluciones
a la solución óptima.
Los buenos resultados obtenidos se deben a la naturaleza de GRASP, que consiste en iterativamente correr una búsqueda local sobre múltiples
soluciones iniciales generadas con un componente aleatorio. El repetir el experimento disminuye la varianza y garantiza una mejor solución
final.

  \newpage

%-- Apéndices --
\section{Apéndices}\label{sec:apendices}
  
  \subsection{Código Fuente (resumen)}\label{subsec:codigo-fuente}
  
\subsubsection{Backtracking}

\definecolor{gris}{rgb}{0.3,0.3,0.3}
\definecolor{rojito}{rgb}{1,0,0}
\lstset{language=C++,tabsize=2,basicstyle=\footnotesize,commentstyle=\color{gris},stringstyle=\color{rojito}}

\begin{lstlisting}[caption=backtracking.cpp]
Timer timer( cerr );

int main() {
    int N;
    while (cin >> N && N) {
        BacktrackingHeuristic b;
        b.parseInput(N);
        timer.setInitialTime( "todo_el_codigo" );
        
        if (b.U != b.V) {
            b.initialize();
            vector<Node> adjacent = b.G->getAdjacent(b.U); // se devuelve por referencia
            for (int i = 0; i < adjacent.size(); i++) {
                Node n = adjacent[i];
                Edge *f = b.G->getEdge(b.U, n);
                b.backtrack(f);
            }
        }
        
        timer.setFinalTime( "todo_el_codigo" );
        timer.saveAllTimes();
        b.printSolution();
    }
    return 0;
}
\end{lstlisting}
\begin{lstlisting}[caption=BacktrackingHeuristic::parseInput()]
void BacktrackingHeuristic::parseInput(int N) {
    this->N = N;
    cin >> this->M >> this->U >> this->V >> this->K;
    this->G = new Graph(N);
    this->visited = vector<bool>(N, false);
    int v1, v2;
    double w1, w2;
    for(int i = 0; i < M; i++) {
        cin >> v1 >> v2 >> w1 >> w2;
        this->G->addEdge(v1, v2, w1, w2);
    }
}
\end{lstlisting}
\begin{lstlisting}[caption=BacktrackingHeuristic::initialize()]
void BacktrackingHeuristic::initialize() {
    this->currentBranch.path.push_back(this->U);
    this->bestSolutionFound.totalOmega2 = INFINITE;
    
    DijkstraSolution byOmega1(N, V);
    DijkstraSolution byOmega2(N, V);
    Dijkstra<ObjectiveFunctionA> dijsktra1; 
    Dijkstra<ObjectiveFunctionB> dijsktra2; 
    dijsktra1.findPath(this->G, &byOmega1);
    dijsktra2.findPath(this->G, &byOmega2);
    this->distancesOmega1 = byOmega1.distances;
    this->distancesOmega2 = byOmega2.distances;
}
\end{lstlisting}
\begin{lstlisting}[caption=BacktrackingHeuristic::backtrack()]
void BacktrackingHeuristic::backtrack(Edge *e) {
    Node toNode = e->toNode;
    this->currentBranch.path.push_back(toNode);
    this->currentBranch.totalOmega1 += e->omega1;
    this->currentBranch.totalOmega2 += e->omega2;
    this->visited[toNode] = true;
    
    bool podar = ((this->currentBranch.totalOmega1 + this->distancesOmega1[toNode]) > this->K) ||
        ((this->currentBranch.totalOmega2 + this->distancesOmega2[toNode]) >= this->bestSolutionFound.totalOmega2);

    if (!podar) {
        if (toNode == this->V) {
            this->bestSolutionFound = currentBranch;
        } else {
            vector<Node> adjacent = G->getAdjacent(toNode); // se devuelve por referencia
            for (int i = 0; i < adjacent.size(); i++) {
                Node n = adjacent[i];
                if (!visited[n]) {
                    Edge *f = this->G->getEdge(toNode, n);
                    backtrack(f);
                }
            }
        }
    }

    this->currentBranch.path.pop_back();
    this->currentBranch.totalOmega1 -= e->omega1;
    this->currentBranch.totalOmega2 -= e->omega2;
    this->visited[toNode] = false;
}
\end{lstlisting}
\begin{lstlisting}[caption=BacktrackingHeuristic::printSolution()]
void BacktrackingHeuristic::printSolution() {
    Solution *s = &(this->bestSolutionFound);
    if (s->totalOmega2 == INFINITE) {
        cout << "no" << endl;
        return;
    }

    cout << s->totalOmega1 << " " << s->totalOmega2 << " " << (s->path.size()+1);
    for (int i = 0; i < s->path.size(); i++)
        cout << " " << s->path[i];
    cout << endl;
    return;
}
\end{lstlisting}
\subsubsection{Greedy}

\begin{lstlisting}[caption=greedy\_heuristic\_All.cpp]
#include <iostream>
#include "../common/Timer.h"
#include "GreedyHeuristicAll.h"

using namespace std;

int main(int argc, char* argv[])
{
  Timer timer(cerr);
  GreedyHeuristicAll greedy(&timer);
  greedy.run(); 
    
    return 0;
}
\end{lstlisting}
\begin{lstlisting}[caption=GreedyHeuristicAll.cpp]
#include "GreedyHeuristicAll.h"
#include "../common/Dijkstra.h"
#include "../common/ObjectiveFunctions.h"

GreedyHeuristicAll::GreedyHeuristicAll()
{
  solution = new Solution();
}

GreedyHeuristicAll::GreedyHeuristicAll(Timer* t)
{
  solution = new Solution();
  timer = t;
}
\end{lstlisting}
\begin{lstlisting}[caption=createSolutionA()]
Solution* createSolutionA(ProblemInstance* instance) 
{    
    // creo el dijkstra
    Dijkstra<ObjectiveFunctionA> dijsktra;
    // creo la solucion
    DijkstraSolution dijkstraSolution( instance->graph->nodeCount, instance->u);
    // cargo en la solucion, todos los paths del dijkstra desde el nodo inicial
    dijsktra.findPath( instance->graph, &dijkstraSolution );
    // obtengo el path que me interesa
    Solution* solution = new Solution();    
    dijkstraSolution.getPath( instance->v, instance->graph, solution->path, solution->totalOmega1, solution->totalOmega2 );
    return solution;
}
\end{lstlisting}
\begin{lstlisting}[caption=createSolutionB()]
Solution* createSolutionB(ProblemInstance* instance) 
{
    // creo el dijkstra    
    Dijkstra<ObjectiveFunctionB> dijsktra;
    // creo la solucion
    DijkstraSolution dijkstraSolution( instance->graph->nodeCount, instance->u);
    // cargo en la solucion, todos los paths del dijkstra desde el nodo inicial
    dijsktra.findPath( instance->graph, &dijkstraSolution );
    // obtengo el path que me interesa
    Solution* solution = new Solution();    
    dijkstraSolution.getPath( instance->v, instance->graph, solution->path, solution->totalOmega1, solution->totalOmega2 );
    return solution;
}
\end{lstlisting}
\begin{lstlisting}[caption=createSolutionC()]
Solution* createSolutionC(ProblemInstance* instance) 
{
    // creo el dijkstra    
    Dijkstra<ObjectiveFunctionC> dijsktra;
    // creo la solucion
    DijkstraSolution dijkstraSolution( instance->graph->nodeCount, instance->u);
    // cargo en la solucion, todos los paths del dijkstra desde el nodo inicial
    dijsktra.findPath( instance->graph, &dijkstraSolution );
    // obtengo el path que me interesa
    Solution* solution = new Solution();    
    dijkstraSolution.getPath( instance->v, instance->graph, solution->path, solution->totalOmega1, solution->totalOmega2 );
    return solution;
}
\end{lstlisting}
\begin{lstlisting}[caption=getBestSolution()]
Solution* getBestSolution(ProblemInstance* instance) 
{
    Solution* solutionB = createSolutionB(instance);    
    if(solutionB->totalOmega1 <= instance->K) {
        return solutionB;
    }
    // si la solucion no es factible probamos con otras funciones objetivo    
    Solution* solutionA = createSolutionA(instance);
    Solution* solutionC = createSolutionC(instance);
    if(solutionC->totalOmega1 <= instance->K) {
        if(solutionA->totalOmega2 < solutionC->totalOmega2) {            
            return solutionA;
        }         
        return solutionC;
    }    
        
    if(solutionA->totalOmega1 <= instance->K) {
        return solutionA;
    }
    return NULL;   
}
\end{lstlisting}
\begin{lstlisting}[caption=GreedyHeuristicAll::resolveInstance()]
void GreedyHeuristicAll::resolveInstance( ProblemInstance* instance ){
  solution = getBestSolution(instance);
}
\end{lstlisting}
\begin{lstlisting}[caption=GreedyHeuristicAll::run()]
void GreedyHeuristicAll::run()
{
  Parser parser;
  parser.parseInput();

  for ( int i = 0; i < parser.problemInstances.size(); i++ )
  {
    ProblemInstance* instance = parser.problemInstances[i];
    
    timer->setInitialTime("todo_el_codigo");
    resolveInstance( instance );       
    timer->setFinalTime("todo_el_codigo");
    timer->saveAllTimes();
    
    if(!solution) {
      cout << "no" << endl;
    } else{
      solution->print();
    }    
  }
}
\end{lstlisting}

\subsubsection{Local Search}

\begin{lstlisting}[caption=local\_search.cpp]
#include "../common/Graph.h"
#include "../common/Parser.h"
#include "../common/DijkstraSolution.h"
#include "../common/Dijkstra.h"
#include "../common/GreedyHeuristic.h"
#include "../common/Solution.h"
#include "../common/Parser.h"
#include "../common/Timer.h"
#include "NeighbourhoodSelectorA.h"
#include "InitialSolution.h"

Parser parser;
Timer timer( cerr );


int main( int argc, char const* argv[] )
{
  /*****************
    Initialization
  ******************/
  // instantiate the initial solution using the initial solution parameter
  InitialSolution* initialSolution = new InitialSolution();
  // instantiate the neighborhood selector using the neighborhood selector parameter
  NeighbourhoodSelector* selector = new NeighbourhoodSelectorA();
  // parse the input
  parser.parseInput();

  /*************************
    Iterate over instances
  *******************'*******/
  for(auto instance:parser.problemInstances)
  {
    /*************
      Resolution
    **************/
    selector->initialize(instance);

    // obtain the initial time
    timer.setInitialTime( "todo_el_codigo" );
    // obtain the initial solution

    int initialSolutionOmega2;

    Solution* solution = initialSolution->getInitialSolution( instance );

    // El dijkstra de omega1 debe cumplir con el K, sino no tiene sentido correr la heuristica
    // si solution no es valida entonces, entonces es NULL
    if(solution != NULL) {
      initialSolutionOmega2 = solution->totalOmega2;

      // run the heuristic
      Solution* newSolution = NULL;
      bool huboMejora = false;
      do
      {
        newSolution = selector->getBestNeighbour( solution );
        // Si no logro mejorar la solucion, termino
        if(newSolution != NULL) {
          delete solution;
          solution = newSolution;          
          huboMejora = true;  
        } else {                      
          huboMejora = false;
        }
      } while(huboMejora);
    } 

    // obtain the final time
    timer.setFinalTime( "todo_el_codigo" );

    /***************
      Output Print
    ****************/
    // print the solution
    if(solution) {
      solution->print();
      delete solution;
    }
    else
    {
      cout << 0 << endl;
    }

    // save all obtained times to output
    timer.saveAllTimes();
  }
  return 0;
}
\end{lstlisting}
\begin{lstlisting}[caption=createSolutionA()]
Solution* createSolutionA(ProblemInstance* instance) 
{    
    // creo el dijkstra
    Dijkstra<ObjectiveFunctionA> dijsktra;
    // creo la solucion
    DijkstraSolution dijkstraSolution( instance->graph->nodeCount, instance->u);
    // cargo en la solucion, todos los paths del dijkstra desde el nodo inicial
    dijsktra.findPath( instance->graph, &dijkstraSolution );
    // obtengo el path que me interesa
    Solution* solution = new Solution();    
    dijkstraSolution.getPath( instance->v, instance->graph, solution->path, solution->totalOmega1, solution->totalOmega2 );
    return solution;
}
\end{lstlisting}
\begin{lstlisting}[caption=createSolutionB()]
Solution* createSolutionB(ProblemInstance* instance) 
{
    // creo el dijkstra    
    Dijkstra<ObjectiveFunctionB> dijsktra;
    // creo la solucion
    DijkstraSolution dijkstraSolution( instance->graph->nodeCount, instance->u);
    // cargo en la solucion, todos los paths del dijkstra desde el nodo inicial
    dijsktra.findPath( instance->graph, &dijkstraSolution );
    // obtengo el path que me interesa
    Solution* solution = new Solution();    
    dijkstraSolution.getPath( instance->v, instance->graph, solution->path, solution->totalOmega1, solution->totalOmega2 );
    return solution;
}
\end{lstlisting}
\begin{lstlisting}[caption=createSolutionC()]
Solution* createSolutionC(ProblemInstance* instance) 
{
    // creo el dijkstra    
    Dijkstra<ObjectiveFunctionC> dijsktra;
    // creo la solucion
    DijkstraSolution dijkstraSolution( instance->graph->nodeCount, instance->u);
    // cargo en la solucion, todos los paths del dijkstra desde el nodo inicial
    dijsktra.findPath( instance->graph, &dijkstraSolution );
    // obtengo el path que me interesa
    Solution* solution = new Solution();    
    dijkstraSolution.getPath( instance->v, instance->graph, solution->path, solution->totalOmega1, solution->totalOmega2 );
    return solution;
}
\end{lstlisting}
\begin{lstlisting}[caption=InitialSolution::getInitialSolution()]
Solution* InitialSolution::getInitialSolution(ProblemInstance* instance)
{   
    Solution* solutionB = createSolutionB(instance);    
    if(solutionB->totalOmega1 <= instance->K) {
        return solutionB;
    }
    // si la solucion no es factible probamos con otras funciones objetivo    
    Solution* solutionA = createSolutionA(instance);
    Solution* solutionC = createSolutionC(instance);
    if(solutionC->totalOmega1 <= instance->K) {
        if(solutionA->totalOmega2 < solutionC->totalOmega2) {
            return solutionA;
        } 
        return solutionC;
    }
    if(solutionA->totalOmega1 <= instance->K) {
        return solutionA;
    }
    return NULL;    
}
\end{lstlisting}
\begin{lstlisting}[caption=NeighbourhoodSelectorA::removeCycles()]
Solution* NeighbourhoodSelectorA::removeCycles(Solution* solution) 
{
  int* nextNodes = new int[nodeCount];
  for(int i=0; i<nodeCount; i++) {
    nextNodes[i] = 0;
  }  
  for(unsigned int i=0; i<solution->path.size(); i++) {
    Edge* edge = solution->path[i];    
    nextNodes[edge->fromNode] = edge->toNode;        
  }        

  vector<int> newPath;
  int firstNode = solution->path[0]->fromNode;
  int lastNode = solution->path[solution->path.size()-1]->toNode;  
  int next = firstNode;
  newPath.push_back(firstNode);     
  int a = 0;
  while(nextNodes[next] != lastNode)
  {    
    next = nextNodes[next];        
    newPath.push_back(next);    
  }

  newPath.push_back(lastNode);
  //cout << lastNode << endl;
  
  Solution* newSolution = new Solution();
  for(int i=0; i<newPath.size()-1; i++){    
    Edge* edge = solution->getEdgeBetween(newPath[i], newPath[i+1]);    
    Edge* newEdge = new Edge(edge->fromNode, edge->toNode, edge->omega1, edge->omega2);
    newSolution->path.push_back(newEdge);
    newSolution->totalOmega1 += newEdge->omega1;
    newSolution->totalOmega2 += newEdge->omega2;
  }

  return newSolution;
}
\end{lstlisting}
\begin{lstlisting}[caption=createNewSolutionReplacingPath()]
Solution* createNewSolutionReplacingPath(const Solution* orig, const Solution* sub) {
  int node1 = sub->path[0]->fromNode;
  int node2 = sub->path[sub->path.size()-1]->toNode;
  Solution* res = new Solution();
  vector<Edge*> edgesToRemove;
  bool subPathStartsAtNode1 = false;
  int subPathStartsAtEdgeIndex = 0;
  // primero agrego todos los edges del path original hasta encontrar alguno de los nodos del subpath
  for(int i=0; i<orig->path.size(); i++) {
    Edge* edge = orig->path[i];
    if(edge->fromNode == node1 || edge->fromNode == node2) {
      subPathStartsAtNode1 = edge->fromNode == node1;
      subPathStartsAtEdgeIndex = i;
      break;
    } else {
      res->path.push_back(edge);
      res->totalOmega1 += edge->omega1;
      res->totalOmega2 += edge->omega2;
    }
  }

  // agrego todos los ejes del sub path
  for(int i=0; i<sub->path.size(); i++) {
    Edge* edge = sub->path[i];
    res->path.push_back(edge);
    res->totalOmega1 += edge->omega1;
    res->totalOmega2 += edge->omega2;
  }

  // busco el nodo desde donde continua el pedazo del path original
  // agrego todos los ejes del path original a partir de ahi
  int fromNode = subPathStartsAtNode1 ? node2 : node1;
  bool addEdges = false;
  for(int i=subPathStartsAtEdgeIndex+1; i<orig->path.size(); i++) {
    Edge* edge = orig->path[i];
    if(edge->fromNode == fromNode) {
      addEdges = true; // encontre el nodo, asi que comienzo a aniadir los nodos desde aca
    }
    if(addEdges) {
      res->path.push_back(edge);
      res->totalOmega1 += edge->omega1;
      res->totalOmega2 += edge->omega2;
    }
  }

  return res;
}
\end{lstlisting}
\begin{lstlisting}[caption=NeighbourhoodSelectorA::getBestNeighbour()]
Solution* NeighbourhoodSelectorA::getBestNeighbour(const Solution* origSolution)
{ 
  // Itero sobre la matriz de soluciones
  // Por cada par de nodos, tengo que buscar si existe un path en la matriz, 
  // tal que sustituyendolo pot path entre el par de nodos de la solucion original 
  // se obtenga una nueva solucion tal que se su totalOmega2 sea menor
  
  // Nota: Habiamos pensado en usar DeltaOmega2, pero si K es muy grande,
  // nos sobraria mucho K que podriamos haber usado para sustituir en cada paso
  // por un path con minimo omega2                
  
  // El path de una solution es un vector de ejes   
  // En cada iteracion que encuentro una solucion mejor, voy a cambiar algunos de los nodos, 
  // por lo que tengo que recalcular los nodos de la mejor solucion             
  int edgesCount = origSolution->path.size();
  int nodesCount = edgesCount+1;
  vector<int> nodes(nodesCount); 
  for(int i=0; i<edgesCount; i++) {
    Edge* edge = origSolution->path[i];
    nodes[i] = edge->fromNode;
  }
  nodes[nodesCount-1] = origSolution->path[edgesCount-1]->toNode; // ultimo nodo del path
      
  Solution* bestSolution = NULL;  
  for(int i=0; i<nodes.size() - 1; i++) {    
    for(int j=i+1; j<nodes.size(); j++) {           
      Solution* subSolution = origSolution->createSubSolutionBetween(nodes[i], nodes[j]); // solution con sub path entre los nodos            
      Solution* solution_ij = getSolvedPathBetween(nodes[i], nodes[j]); // dijkstra por omega2 entre los nodos
      if(solution_ij == NULL) {
        continue;
      }      
      
      // Si el path creado con dijkstra usando omega2, tiene menos omega2 total, que el path actual entre
      // los nodos i y j, entonces chequeo si al crear una nueva solucion tendra menos omega2 que la mejor solucion.
      // En ese caso me guardo esta nueva solucion como la mejor hasta ahora.
      // Ademas chequeo que se cumpla con el K requerido.                          
      double bestSolutionOmega2 = bestSolution == NULL ? origSolution->totalOmega2 : bestSolution->totalOmega2;
      double newSolutionOmega2 = origSolution->totalOmega2 - subSolution->totalOmega2 + solution_ij->totalOmega2;
      double newSolutionOmega1 = origSolution->totalOmega1 - subSolution->totalOmega1 + solution_ij->totalOmega1;            
      if(newSolutionOmega2 < bestSolutionOmega2 && newSolutionOmega1 <= K) {                     
        if(bestSolution != NULL) {
          delete bestSolution;   
        }       
        // creo la nueva mejor solucion        
        bestSolution = createNewSolutionReplacingPath(origSolution, solution_ij);           
        Solution* bestSolutionWithoutCycles = removeCycles(bestSolution);                         
        delete bestSolution;
        bestSolution = bestSolutionWithoutCycles;        
      }      
      delete subSolution;
    }
  }
    
  return bestSolution;  
}
\end{lstlisting}
\begin{lstlisting}[caption=NeighbourhoodSelector::initialize()]
void NeighbourhoodSelector::initialize(ProblemInstance* instance)
{	
	deleteMatrix();
	nodeCount = instance->graph->nodeCount;
	pathMatrix = new Solution*[nodeCount*nodeCount];     
  K = instance->K;   

  // Como los paths entre j y k son iguales a los paths entre k y j, pero al reves
  // primero encuentro los paths entre j y k y luego los doy vuelta
	for(int j=1; j<=nodeCount; j++) {
  	Dijkstra<ObjectiveFunctionOmega2> dijsktra;          
    DijkstraSolution dijkstraSolution( instance->graph->nodeCount, j );                
    dijsktra.findPath( instance->graph, &dijkstraSolution );                      
    for(int k=1; k<=nodeCount; k++) {                    
      if(j==k) continue;
    	  Solution* solution = new Solution();        
    	  dijkstraSolution.getPath(k, instance->graph, solution->path, solution->totalOmega1, solution->totalOmega2);          
        // si no hay ninun path entre los nodos j y k entonces, cargo NULL en la matrix             
        pathMatrix[j-1 + (k-1) * nodeCount] = solution->path.size() > 0 ? solution : NULL;                                  
    }
	}
}
\end{lstlisting}
\begin{lstlisting}[caption=NeighbourhoodSelector::getSolvedPathBetween()]
Solution* NeighbourhoodSelector::getSolvedPathBetween(int node1, int node2) {  
  return pathMatrix[node1-1 + (node2-1) * nodeCount];
}
\end{lstlisting}
\subsubsection{Grasp}

\begin{lstlisting}[caption=grasp.cpp]
#include "../common/Graph.h"
#include "../common/Parser.h"
#include "../common/DijkstraSolution.h"
#include "../common/Dijkstra.h"
#include "../common/DijkstraRandomized.h"
#include "../common/GreedyHeuristic.h"
#include "../common/Solution.h"
#include "../common/Parser.h"
#include "../common/Timer.h"
#include "NeighbourhoodSelectorA.h"
#include "InitialSolution.h"
#include <math.h>

Parser parser;
Timer timer( cerr );


int main( int argc, char const* argv[] ) 
{
    /* Semilla random */
    srand (time(NULL));

    /*****************
      Initialization
    ******************/
            
    // parse the input
    parser.parseInput();  

    // obtain the initial time
    timer.setInitialTime( "todo_el_codigo" );
    // obtain the initial solution

    /*************************
      Iterate over instances
    **************************/
    for(auto instance:parser.problemInstances)
    {        
        /*************
         Resolution
        **************/           
        NeighbourhoodSelector* selector = new NeighbourhoodSelectorA();
        selector->initialize(instance);                        

        // valores arbitrarios basados en n para criterio de terminaciones
        int n = instance->graph->nodeCount;
        int iteracionesSinMejorarCount = 0;
        int iteracionesSinMejorarMax = n*n/4;
        int iteracionesMax = n*n/2;
        int iteracionesSinInitialPathCount = 0;
        int iteracionesSinInitialPathMax = n*n/4;
        
        Solution* bestSolution = NULL;            

        for(int i = 0; i<iteracionesMax; i++) {              
            // instantiate the initial solution using the initial solution parameter
            InitialSolution* initialSolution = new InitialSolution();              
            Solution* solution = initialSolution->getInitialSolution( instance );                     
            if(solution->path.size() == 0) {
                // no encontre un path entre u y v
                iteracionesSinInitialPathCount++;
                delete solution;            
                if(iteracionesSinInitialPathCount < iteracionesSinInitialPathMax) {
                    continue; // sigo intentando buscar soluciones
                } else {
                    break; // me rindo, dejo de buscar soluciones
                }            
            }                        

            // El dijkstra de omega1 debe cumplir con el K, sino no tiene sentido correr la heuristica            
            if(solution->totalOmega1 <= instance->K) {                               
                // run the heuristic
                Solution* newSolution = NULL;    
                bool huboMejora = false;
                do
                {                                        
                    newSolution = selector->getBestNeighbour( solution );                                                            
                    // Si no logro mejorar la solucion, termino      
                    if(newSolution != NULL) {                        
                        //cout << newSolution->totalOmega2 << endl; 
                        delete solution;
                        solution = newSolution;          
                        huboMejora = true;                          
                    } else {                      
                        huboMejora = false;
                    }        
                } while(huboMejora);

                if(bestSolution == NULL) {
                    bestSolution = solution;
                } else if(solution->totalOmega2 < bestSolution->totalOmega2) {
                    delete bestSolution;
                    bestSolution = solution;
                } else {
                    delete solution;
                    iteracionesSinMejorarCount++;      
                }            
            }       

            if(iteracionesSinMejorarCount > iteracionesSinMejorarMax) {      
                break;
            }
        }  

        // obtain the final time
        timer.setFinalTime( "todo_el_codigo" );

        /***************
          Output Print
        ****************/
        // print the solution
        if(bestSolution != NULL) {
            bestSolution->print();
            delete bestSolution;
        } else {
            cout << "no" << endl;
        }    
        
        // save all obtained times to output
        timer.saveAllTimes();
    }    
    
    return 0;
}
\end{lstlisting}
\begin{lstlisting}[caption=InitialSolution::getInitialSolution()]
Solution* InitialSolution::getInitialSolution(ProblemInstance* instance)
{    
  // creo el dijkstra
    DijkstraRandomized<ObjectiveFunctionA> dijsktraRandom;
    // creo la solucion
    DijkstraSolution dijkstraSolution( instance->graph->nodeCount, instance->u);
    // cargo en la solucion, todos los paths del dijkstra desde el nodo inicial
    dijsktraRandom.findPath( instance->graph, &dijkstraSolution );
    // obtengo el path que me interesa
    Solution* solution = new Solution();    
    dijkstraSolution.getPath( instance->v, instance->graph, solution->path, solution->totalOmega1, solution->totalOmega2 );
    return solution;
}
\end{lstlisting}

\begin{lstlisting}[caption=DijkstraRandomized::findPath()]
 void DijkstraRandomized<ObjectiveFunction>::findPath(Graph* graph, DijkstraSolution* solution) {
    int* prevNodes = solution->prevNodes;
    double* dist = new double[graph->nodeCount];
    for (int i=0; i<graph->nodeCount; i++) {
        dist[i] = INF;
        prevNodes[i] = -1;
    }
    dist[solution->fromNode-1] = 0;

    ObjectiveFunction objFunc;

    unvisited.push(new UnvisitedNode(solution->fromNode, 0, 0));
    
    while (unvisited.size() > 0) {
        int rcl_size = min(RCL_SIZE, int(unvisited.size()));
        int chosen = rand()%rcl_size + 1; // genera un int entre 1 y rcl_size         
        list<UnvisitedNode*> stack;
        for (int i = 0; i < chosen; i++) {
            stack.push_front(unvisited.top());
            unvisited.pop();
        }
        UnvisitedNode* currNode = stack.front();        
        stack.pop_front();
        for (list<UnvisitedNode*>::iterator it = stack.begin(); it != stack.end(); it++)
            unvisited.push(*it);
                
        vector<Node> adjNodes = graph->getAdjacent(currNode->node);        
        for (int i=0; i<adjNodes.size(); i++) {            
            Node toNode = adjNodes[i];
            if(prevNodes[toNode-1] != -1) continue;
            Edge* edge = graph->getEdge(currNode->node, adjNodes[i]);
            double weight = objFunc.weight(edge->omega1, edge->omega2);
            if(dist[currNode->node-1] + weight < dist[toNode-1]) {
                prevNodes[toNode-1] = currNode->node;
                dist[toNode-1] = dist[currNode->node-1] + weight;                
                unvisited.push(new UnvisitedNode(toNode, edge->omega1, edge->omega2));
            }
        }
    }    
}
\end{lstlisting}

\begin{lstlisting}[caption=DijkstraSolution::getPath()]
void DijkstraSolution::getPath(int toNode, Graph* graph, vector<Edge*> &path, double &totalOmega1, double &totalOmega2) {	
    int prevNode = prevNodes[toNode-1];
    totalOmega1 = 0;
    totalOmega2 = 0;        
    list<Edge*> pathList;
    while (prevNode != -1) {
    	Edge* edge = graph->getEdge(prevNode, toNode);
        pathList.push_front(new Edge(prevNode, toNode, edge->omega1, edge->omega2));
        totalOmega1 += edge->omega1;
        totalOmega2 += edge->omega2;
        toNode = prevNode;
        prevNode = prevNodes[prevNode-1];
    }        

    // toNode va cambiando dentro del while
    // si llegado a este punto, toNode != fromNode, significa que no hay camino entre toNode y fromNode    
    if(toNode != fromNode) {
        totalOmega1 = INF;
        totalOmega2 = INF;
        // el path que se devuelve en la solucion queda vacio
    } else {
        // solo devuelvo un camino si existe un camino posible entre fromNode y toNode
        path.resize(pathList.size());
        int index = 0;
        for(list<Edge*>::iterator it = pathList.begin(); it != pathList.end(); it++) {
            path[index] = *it;
            index++;
        }
    }
} 
\end{lstlisting}

  \clearpage
  
  %
  % - Para la reentrega -
  %
  %\subsection{Informe de Modificaciones}
  %\begin{itemize}
    \item Se revisaron en general todas las secciones de acuerdo a las correcciones.
    \item Se agregaron dos nuevas secciones: \textit{Pautas generales de medición} y \textit{Pautas para la generación de grafos}.
\end{itemize}

  %\clearpage
  
  %-- Bibliografia --
  %\subsection{Bibliografía}
  %\begin{thebibliography}{99}
% 
%   \bibitem{lib:brassard} G. Brassard, P. Bratley, \textit{Fundamental of Algorithmics}, Prentice Hall,  1996, Chapter 6.2, \enquote{General characteristics of greedy algoritmhs}, Chapter 9.6, \enquote{Backtracking}, 
%   \bibitem{lib:cormen} Cormen, Leiserson, Rivest \textit{Introduction to Algorithms}, 2001, Chapter 16 \enquote{Greedy Algorithms}.
%   \bibitem{stl:stl} \texttt{http://www.cplusplus.com/reference/stl/}
%   \bibitem{stl:set} \texttt{http://www.cplusplus.com/reference/set/set/}
%   \bibitem{stl:vector} \texttt{http://www.cplusplus.com/reference/vector/vector/}
%   \bibitem{stl:pqueue} \texttt{http://www.cplusplus.com/reference/queue/priority\_queue/}
%   \bibitem{wiki:greedy} \texttt{http://en.wikipedia.org/wiki/Greedy\_algorithm\#Specifics}
%   \bibitem{wiki:back} \texttt{http://en.wikipedia.org/wiki/Backtracking\#Usage\_considerations}

  %\end{thebibliography}

\end{TP3}
\end{document}
