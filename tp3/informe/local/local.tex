
% Búsqueda local
La búsqueda local es un método heurístico para encontrar una solución factible a un problema. Se basa en obtener una solución inicial, $s$, y luego mejorar esa solución iterativamente, tomando la mejor solución de un conjunto de vecinos de $s$. El conjunto de los vecinos se determina mediante algún criterio y se usa una función objetivo para comparar las soluciones en la vecindad de $s$ y discernir cual es la mejor.

La búsqueda local se puede expresar de la siguiente manera:\\\\
\hspace*{1 cm} Sea $s \in S$ una solución inicial\\
\hspace*{1 cm} Mientras exista $s' \in N(s)$ con $f(s') > f(s)$:\\
\hspace*{2 cm} $s \leftarrow s'$

Siendo $N(s)$ la vecindad de $s$ y $f$ la función objetivo.

% Solución inicial
\subsubsection{Solución inicial}

Para obtener una solución inicial utilizamos Dijkstra. Experimentamos con la funciones objetivo Greedy A y Greedy C descritas en el apartado anterior(ver sección de heurística golosa). No utilizamos Greedy B, ya que de encontrar una solución usando la sumatoria de los pesos $\omega_2$ como función objetivo, podía pasar que no encontremos una solución o que encontremos la solución y en ese caso, no tiene sentido usar búsqueda local, ya que no hay mejor solución posible.

En caso de usar Greedy C y no obtener una solución inicial factible, corremos Dijkstra nuevamente, pero utilizando la sumatoria de los pesos $\omega_1$ como función objetivo(o sea, Greedy A). La razón de usar esa función objetivo es que se minimiza el $\omega_1$ total del camino encontrado y por ende hay una chance máxima de obtener una solución que cumpla con la cota K. Eso es porque Dijkstra encuentra el camino mínimo de uno nodo hacia todos los demás\footnote{Demostrado en la sección de heurística golosa}, y si no se encuentra un camino con $\omega_1$ mínimo que cumpla con la cota K, significa que no existe ningún otro camino que cumpla con la cota. En caso de que eso suceda, podemos asegurar que no existe solución al problema.

% Definición de la vecindad
\subsubsection{Definición de la vecindad}

En cada iteración definimos la vecindad de $s$, $N(s)$, de la siguiente manera: dado el grafo inicial G, y una solución formada por un camino $c \in G$, definimos sus soluciones vecinas como aquellas resultantes de tomar un subcamino $d_{n_1,n_2} \in c$ entre un par de nodos $n_1,n_2$ cualesquiera, y reemplazarlo por otro camino $d_{n_1,n_2}^* \in G$, de tal forma que el camino $c^* = c - d_{n_1,n_2} + d_{n_1,n_2}^*$ resultante cumpla: 

\begin{itemize}
\item $\omega_1(c^*)$ $<$ $K$
\item $\omega_2(c^*)$ $<$ $\omega_2(c)$
\end{itemize}

De la forma descripta, dada una solución $S$ formada por un camino $c$ definimos su vecindad como el conjunto $S^*$ de todos los caminos $c^*$ posibles.

Para obtener el camino $d_{n_1,n_2}^*$ utilizamos Dijkstra con la sumatoria de los pesos $\omega_2$ como función objetivo(Greedy B). Lo que buscamos es mejorar el subcamino $d_{n_1,n_2} \in c$ obteniendo otro camino que tenga menor $\omega_2$, y dado que usamos Dijkstra para encontrar el camino con $\omega_2$ mínimo entre los nodos, entonces estamos obteniendo un camino que va a tener igual o menor $\omega_2$. En caso de que sea posible hacer el reemplazo, como disminuimos el $\omega_2$ de una parte del camino y dejamos el resto del camino igual, estamos logrando disminuir el $\omega_2$ del camino completo.

% Selección de vecino
\subsubsection{Selección de vecino}

Dada la vecindad $S^*$, se elige al vecino usando Steepest descent, con $\omega_2$ como función objetivo.

\subsubsection{Pseudocódigo}

El algoritmo está implementado en la función \texttt{main}:

\begin{algorithm}[H]
\caption{$main$(int tipo\_solucionInicial, Graph g, Nodo n1, Nodo n2)}
\begin{algorithmic}[1]
  \State crearMatrizCaminosMinimos(g)
  \State Solution solucion = obtenerSolucionInicial(tipo\_solucionInicial, g, n1, n2)
  \If{tipo\_solutionInicial $\neq$ Greedy\_A \&\& $\omega_1$(solucion) $>$ K}
    solucion = obtenerSolucionInicial(Greedy\_A, g, n1, n2)
  \EndIf
  
  \If{$\omega_1$(solucion) $\leq$ K}    
    \While{True}    	        
    	\State Solution nuevaSolucion = dameMejorVecino(solucion)
	\If{nuevaSolucion == NULL} 
		\State break	
	\EndIf    
	\State solucion = nuevaSolucion	
    \EndWhile
  \EndIf
\end{algorithmic}
\end{algorithm}

\begin{algorithm}[H]
\caption{$obtenerSolucionInicial$(int tipo, Graph g, Nodo n1, Nodo n2)}
\begin{algorithmic}[1]
  \If{tipo == Greedy\_C}
	\State return resolverConDijkstra(g, n1, n2, ObjectiveFunctionC)
  \EndIf
  \State return resolverConDijkstra(g, n1, n2, ObjectiveFunctionA)
\end{algorithmic}
\end{algorithm}

\begin{algorithm}[H]
\caption{$dameMejorVecino$(Solution solucionOriginal)}
\begin{algorithmic}[1]	
	  \State Solution mejorSolucion = solucionOriginal
	  \State vector<int> nodos = nodos(solucionOriginal)
	  \For{i=0; i<size(nodos); i++}
	  	\For{j=i+1; j<size(nodos); j++}
			\State Solution subSolucion = crearSubSolucionEntre(solucionOriginal, nodos[i], nodos[j])
			\If{subSolucion == NULL}
				\State continue
			\EndIf
			\State Solution solucion\_ij = dameCaminoResueltoEntre(nodos[i], nodos[j])
			\State Solution nuevaSolucion\_$\omega_2$ = $\omega_2$(solucionOriginal) - $\omega_2$(subSolucion) + $\omega_2$(solucion\_ij)
			\State Solution nuevaSolucion\_$\omega_1$ = $\omega_1$(solucionOriginal) - $\omega_1$(subSolucion) + $\omega_1$(solucion\_ij)			
			\If{nuevaSolucion\_$\omega_2$ $<$ $\omega_2$(mejorSolucion) \&\& nuevaSolucion\_$\omega_1$ $<$ K)}
				\State mejorSolucion = crearSolucionReemplazandoCamino(solucionOriginal, solucion\_ij)
			\EndIf
		\EndFor
	\EndFor
	\State return mejorSolucion
\end{algorithmic}
\end{algorithm}

\begin{algorithm}[H]
\caption{$crearSolucionReemplazandoCamino$(Solution orig, Solution sub)}
\begin{algorithmic}[1]	
	 \State Solution res
	 \State res = obtenerCaminoHasta(nodos(sub)[0])
	 \State res += sub
	 \State int subSize = size(nodos(sub))
	 \State res += obtenerCaminoDesde(nodos(sub)[subSize-1])	  
	\State return res
\end{algorithmic}
\end{algorithm}

% Complejidad
\subsubsection{Complejidad}

Implementamos el algoritmo en la función $main$. La complejidad del algoritmo resulta la suma de obtener la solución inicial y el ciclo que se usa para mejorarla buscando un mejor vecino en cada iteración. 
Además, en $main$, inicialmente, se llama a una función crearMatrizCaminosMinimos que abordaremos más adelante.

Para obtener la solución inicial, usamos obtenerSolucionInicial. Esta función devuelve un camino mínimo entre 2 nodos llamando a resolverConDijkstra con alguna función objetivo(Greedy\_A o Greedy\_C). La función resolverConDijkstra primero ejecuta Dijkstra para encontrar todos los caminos mínimos entre n1 y los demás nodos y después hace un traceback desde n2 hasta n1 para dar con el camino mínimo entre ellos. Como se comprobó en el apartado anterior, la complejidad de Dijkstra es $O(m log(n))$ y el traceback es $O(n)$, por lo que en total la complejidad de obtenerSolucionInicial es $O(m log(n))$. Cabe notar que la función objetivo usada en Dijkstra no influye en la complejidad, ya que solo comparara los valores de $\omega_1$ y $\omega_2$ y por lo que tiene complejidad $O(1)$.

Para mejorar la solución usamos un ciclo y en cada iteración obtenemos el mejor vecino del camino actual usando dameMejorVecino. Ejecutamos el ciclo mientras que hayamos encontrado una mejora al camino actual en la última iteración. 
Dado que un vecino consiste en reemplazar la porción del camino actual que une 2 nodos $n_1$ y $n_2$ por el camino mínimo entre ellos, y que cada vez que se hace el reemplazo se disminuye $\omega_2$ del camino actual, se pueden hacer a lo sumo tantos reemplazos como caminos mínimos entre todo par de nodos del grafo existan. 
La cantidad de caminos mínimos entre todo par de nodos de un grafo es $n * (n-1) / 2$, ya que cada nodo tiene un camino mínimo hacia todos los demás y no a sí mismo. Esto es, si hay n nodos, el nodo $n_1$, tiene un camino mínimo hacia los nodos $n_2$, ..., $n_n$, el nodo $n_2$ tiene un camino hacia $n_3$, ..., $n_n$ ya que no se vuelve a contar el camino entre $n_1$ y $n_2$ y así hasta $n_{n-1}$ que tiene un camino hasta $n_n$.
Luego, la cantidad máxima de iteraciones es $n * (n-1) / 2$.

Para analizar la complejidad de cada iteración hay que analizar la complejidad de dameMejorVecino. 
Lo primero que hace la función es obtener el arreglo de nodos de la solución pasada por parámetro, que llamamos solucionOriginal. Cómo el camino de solucionOriginal está representado como una lista de ejes, lo que hace es iterar por todos los ejes y tomar el nodo1 del eje y al final adicionar el nodo2 del último eje. Por ende, tomando $t$ como la cantidad de nodos en el camino, en cada iteración, por cada eje del camino, se toma un nodo y esto se hace $t$-1 veces y luego se añade el nodo final. Como $t$ puede ser a lo sumo $n$, entonces la complejidad de esto resulta $O(n^2)$.
Luego de ejecuta un doble $for$ de $t * (t-1) / 2$ iteraciones, que en cada iteración intenta mejorar el mejor camino encontrado hasta el momento, que llamamos mejorSolucion. Inicialmente mejorSolucion es igual a solucionOriginal. 
Para encontrar una mejor solución en cada iteración hacemos lo siguiente:

\begin{itemize}
\item Creamos el subcamino de solucionOriginal entre los nodos $n_1$ y $n_2$.
\item Obtenemos el camino mínimo entre los nodos $n_1$ y $n_2$.
\item Obtenemos un nuevo camino reemplazando el el subcamino entre los nodos $n_1$ y $n_2$ por el camino mínimo entre ellos.
\end{itemize}

Para crear el subcamino de solucionOriginal entre $n_1$ y $n_2$ usamos crearSubSolucionEntre. Esta función recibe un par de nodos y un camino y devuelve el subcamino que une a los nodos. Para esto tiene que recorrer a lo sumo todos los nodos del camino, o sea $t$, que puede ser a lo sumo $n$.

Para obtener el camino mínimo entre los nodos $n_1$ y $n_2$ usamos una optimización que es que en la función $main$, al comienzo, ejecutamos crearMatrizCaminosMinimos que crea una matriz de caminos mínimos entre todo par de nodos del grafo. Generamos esta matriz ejecutando Dijkstra para todos los nodos usando como función objetivo $\omega_2$. Usamos esta matriz para obtener en $O(1)$ el camino mínimo entre los nodos $n_1$ y $n_2$.
Como crearMatrizCaminosMinimos ejecuta un Dijkstra por cada nodo, su complejidad es $O(n * (m log n))$.

Para obtener el nuevo camino reemplazando el subcamino entre $n_1$ y $n_2$ por el camino mínimo entre ellos, usamos crearSolucionReemplazandoCamino.
Lo que hacemos es generar una nueva solución que es la concatenación de 3 caminos: el camino mínimo entre $n_1$ y $n_2$ y los dos pedazos del camino original sin el subcamino que unía $n_1$ y $n_2$. Para crear el camino, hay que recorrer el solucionOriginal y añadir todos los nodos hasta $n_1$ y luego añadir todos los nodos del camino mínimo y al final añadir todos los nodos desde $n_2$ hasta el final de solucionOriginal. Por ende es como iterar sobre el nuevo camino que a lo sumo puede tener $n$ nodos y por ende la complejidad resulta $O(n)$.

Entonces, resulta ser que encontrar el mejor vecino, consiste en ejecutar el doble $for$ de algo que tiene complejidad $O(n + 1 + n)$, o sea $O(n)$, entonces en total es $O(n^2 * n) = O(n^3)$.

Como habíamos explicado que dameMejorVecino se ejecuta en un ciclo de hasta $n * (n-1) / 2$ iteraciones, o sea $O(n^2)$, y entonces resulta que la complejidad de encontrar el mejor vecino es $O(n^2 * n^3) = O(n^5)$. 

Finalmente la complejidad total resulta ser la suma de las complejidades de crearMatrizCaminosMinimos, 2 veces obtenerSolucionInicial y $n^2$ veces dameMejorVecino. Es decir, $O(n * (m log n) + 2 * (m log n) + n^5) = O(n^5)$.

% Familias malas
\subsubsection{Familias malas}

Si elegimos como solución inicial a la heurística Greedy B en la sección previa, se ha expuesto que puede fallar en el intento de dar una solución factible, aún existiendo una. El Greedy A siempre encuentra una solución factible de existir ésta. 

Veamos que tomando nuestra heurística de búsqueda local como solución inicial, puede quedar arbitrariamente lejos de la solución óptima.

Presentamos la siguiente familia de grafos:

\ponerGrafico{imagenes/maloLocalA.png}{}{0.5}{malo-para-local-a}

Para ir de 1 a 5 hay tres caminos posibles: ($C_1$) $1 \rightarrow 2 \rightarrow 5$; ($C_2$) $1 \rightarrow 3 \rightarrow 5$;
($C_3$) $1 \rightarrow 4 \rightarrow 5$;

\begin{eqnarray}
 \omega_1(C_1) &=& 4	\\ 
 \omega_2(C_1) &=& 2	\\
 \omega_1(C_2) &=& 2	\\
 \omega_2(C_2) &=& 2x   \\
 \omega_1(C_3) &=& 6	\\
 \omega_2(C_3) &=& 1
\end{eqnarray}

Supongamos que K vale 4.
Como decíamos, partimos del camino mínimo entre $u$ y $v$ de acuerdo a $\omega_1$, $C_2$. El algoritmo va a intentar intercambiar $C_2$ por
$C_3$, el camino que minimiza $\omega_2$. Sin embargo, como éste se pasa del límite K, el algoritmo no puede seguir y devuelve $C_2$. Sin
embargo $C_1$ era una solución mejor.

$\frac{\omega_2(C_2)}{\omega_2(C_1)} = x$.

Haciendo crecer el valor de $x$ podemos encontrar grafos en los que nuestro algoritmo devuelve una
solución arbitrariamente lejos de la óptima.

