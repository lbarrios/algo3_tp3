
Para resolver un problema, un algoritmo goloso sigue una heur\'istica que consiste elegir en cada paso, entre un conjunto de opciones, una soluci\'on \'optima local, esperando encontrar al final la soluci\'on \'optima global. En general estos algoritmos son eficientes y simples de dise\~nar e implementar, pero puede ser que nunca lleguen a la soluci\'on \'optima del problema. 

De acuerdo a la definici\'on de Brassard\footnote{\label{Brassard}Brassard G., Bratley P., {\it Fundamental of Algorithmics}, Prentice Hall, 1996. (c)}, un algoritmo goloso se compone de los siguientes elementos: 

\begin{enumerate}
 \item Un conjunto de candidatos que ya han sido considerados y seleccionados. 
 \item Un conjunto de candidatos considerado y rechazados. 
 \item Una funci\'on que comprueba si cierto conjunto de candidatos constituye una soluci\'on a nuestro problema, ignorando si es o no \'optima por el momento. 
 \item Una funci\'on de factibilidad, que me dice si es posible o no completar el conjunto a\~nadiendo otros candidatos para obtener al menos una soluci\'on de nuestro problema. 
 \item Una funci\'on selecci\'on que indica en cualquier momento cu\'al es el m\'as prometedor de los candidatos restantes, que no han sido seleccionados ni rechazados. 
 \item Una funci\'on objetivo, que da el valor de la soluci\'on que hemos hallado. 
\end{enumerate}

Lo que busca el algoritmo goloso es encontrar el conjunto de candidatos que constituya una soluci\'on, y que optimice el valor de la funci\'on objetivo. Este algoritmo avanza paso a paso. Inicialmente, el conjunto de elementos seleccionados est\'a vac\'io. Entonces, en cada paso se considera a\~nadir a este conjunto el mejor candidato sin coniderar los restantes, de acuerdo a nuestra funci\'on selecci\'on. Si el conjunto ampliado de candidato seleccionados ya no fuera factible, rechazamo el candidato que estamos considerando en ese momento. Sin embargo, si el conjunto aumentado sigue siendo factible, entonces a\~nadimos el candidato actual al conjunto de candidatos selecccionados, en donde pasar\'a a estar desde ahora en adelante. Cada vez que se ampl\'ia el conjunto de candidatos seleccionados, comprobamos si \'este constituye ahora una soluci\'on para nuestro problema. A partir de este esquema, al agregar siempre subsoluciones \'optimas a mi conjunto, al finalizar lo que se espera encontrar es la soluci\'
on \'optima. 

El algoritmo de Dijsktra para encontrar caminos m\'inimos en un grafo pesado es un algoritmo goloso, que funciona y es correcto, como lo fue demostrado por Bassard en el libro mencionado.

Dado un grafo $G=(V,X)$, Dijkstra guarda un conjunto $S$ de nodos que ya fueron recorridos y un vector $\pi$ con la distancia m\'inima de un nodo $u$ a todos los de $S$. En cada fase de Dijkstra, se selecciona un nuevo nodo de $V\backslash S$ cuyo valor en $\pi$ sea m\'inima y lo a\~nadimo a $S$, actualizando si es necesario $\pi$. Al finalizar, $\pi$ es el vector con la m\'inima distancia a todos los nodos. 

Entonces, nosotros para resolver el problema vamos implementar Dijkstra con tres funciones objetivo diferentes, que toman una arista y devuelven un peso para ella: 

\begin{enumerate}
  \item $f_A(e) = \omega_1(e)$
  \item $f_B(e) = \omega_2(e)$
  \item $f_C(e) = \omega_1(e)\omega_2(e)$
\end{enumerate}

Luego el pseudoc\'odigo de Dijkstra modificado con la nueva definici\'on de distancia queda dado por: 

\begin{algorithm}
In: Grafo $G = (V,X)$, nodo inicial $v_0$, ObjectiveFunction $f$ \newline
Out: Arreglo $\pi$ con camino m\'inimo en funci\'on de $f$ a cada nodo. 
\begin{algorithmic}[1]
\State $\pi(v) = \infty$ \quad $\forall v \in V$
\State $\pi(v_0) = 0$
\State $S = \emptyset$
\For{$i = 1 \dots n-1$}
    \State $v \leftarrow $ nodo de $V\backslash S$ de m\'inimo $\pi$. 
    \For{{\bf each} $w \in V\backslash S$ adyacente a $v$}
      \State $\pi(w) = \min( \pi(w), \pi(v) + f((v,w)))$
    \EndFor
    \State $S = S \cup \{v\}$
\EndFor
\State \textbf{retornar} $\pi$
\end{algorithmic}
\end{algorithm}

La modificaci\'on est\'a la l\'inea 7, que en vez de sumar a $\pi(v)$ el peso de la arista, como es en el algoritmo original, le suma el valor de una funci\'on que define el peso de la arista. Esto nos permite mucha flexibilidad a la hora de cambiar la ``decisi\'on golosa''.

%-- Goloso A --
\clearpage
\subsubsection{Greedy A}\label{subsubsec:greedy-a}
Dado un grafo $G = (V,E)$, obtenemos el camino m\'inimo entre $u$ y $v$ seg\'un $\omega_1$. 

Familias malas
\ponerGrafico{imagenes/maloGreedyA.png}{}{0.5}{malo-para-greedy-a}

Para ir de 1 a 4 hay dos caminos posibles: ($C_1$) $1 \rightarrow 2 \rightarrow 4$; ($C_2$) $1 \rightarrow 3 \rightarrow 4$

\begin{eqnarray}
 \omega_1(C_1) &=& 2 	\\ 
 \omega_2(C_1) &=& 2x	\\
 \omega_1(C_2) &=& 2x	\\
 \omega_2(C_2) &=& 2
\end{eqnarray}

%-- Goloso B --
\clearpage
\subsubsection{Greedy B}\label{subsubsec:greedy-b}
Dado un grafo $G = (V,E)$, obtenemos el camino m\'inimo entre $u$ y $v$ seg\'un $\omega_2$. 

\clearpage
%-- Goloso C --
\subsubsection{Greedy C}\label{subsubsec:greedy-c}
Dado un grafo $G = (V,E)$, obtenemos el camino m\'inimo entre $u$ y $v$ seg\'un $\omega_1\omega_2$. 
