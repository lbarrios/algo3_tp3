
\subsubsection{Backtracking}

\small
\begin{verbatim}
int N, M, U, V, K;
Graph *G;
vector<bool> visitados;
Solucion mejorSolucion;
Solucion ramaActual;

void backtrack(Node actual, Node padre) {
  // primero agrego el nodo
  ramaActual.camino.push_back(actual);

  Edge *e = G->getEdge(padre, actual);

  if (actual != padre) {  // me cubro del nodo inicial
    ramaActual.sumaOmega1 += e->omega1;
    ramaActual.sumaOmega2 += e->omega2;
  }
  ramaActual.cantAristas++;
  visitados[actual] = true;
  
  if (ramaActual.sumaOmega1 > K) {
    // no haces nada
  } else if (actual == V) { 
    if (ramaActual.sumaOmega2 < mejorSolucion.sumaOmega2)
      mejorSolucion = ramaActual;
  } else { // llamas a la recursion
    vector<Node> vecinos = G->getAdjacent(actual);
    for (int i = 0; i < vecinos.size(); i++) {
      Node vecino = vecinos[i];
      if (! visitados[vecino]) {
        backtrack(vecino, actual);
      }
    }
  }
  // dejas todo como estaba, antes de retornar
  ramaActual.camino.pop_back();
  if (actual != padre) {
    ramaActual.sumaOmega1 -= e->omega1;
    ramaActual.sumaOmega2 -= e->omega2;
  }
  ramaActual.cantAristas--;
  visitados[actual] = false;
}
\end{verbatim}
\normalsize

\subsubsection{Greedy}

\small
\begin{verbatim}
void DijkstraSolution::getPath(int toNode, Graph* graph, vector<Edge*> &path, 
                                            double &totalOmega1, double &totalOmega2) {           
  int prevNode = prevNodes[toNode-1];
  totalOmega1 = 0;
  totalOmega2 = 0;    
  list<Edge*> pathList;
  while (prevNode != -1) {
             Edge* edge = graph->getEdge(prevNode, toNode);
    pathList.push_front(new Edge(prevNode, toNode, edge->omega1, edge->omega2));
    totalOmega1 += edge->omega1;
    totalOmega2 += edge->omega2;
    toNode = prevNode;
    prevNode = prevNodes[prevNode-1];
  }    

  // toNode va cambiando dentro del while
  // si llegado a este punto, toNode != fromNode, significa que no hay camino 
  // entre toNode y fromNode  
  if(toNode != fromNode) {
    totalOmega1 = INF;
    totalOmega2 = INF;
    // el path que se devuelve en la solucion queda vacio
  } else {
    // solo devuelvo un camino si existe un camino posible entre fromNode y toNode
    path.resize(pathList.size());
    int index = 0;
    for(list<Edge*>::iterator it = pathList.begin(); it != pathList.end(); it++) {
      path[index] = *it;
      index++;
    }
  }
}

template<class ObjectiveFunction>
void GreedyHeuristic<ObjectiveFunction>::resolveInstance( ProblemInstance* instance ){
  // creo el dijkstra
  Dijkstra<ObjectiveFunction> dijsktra;
  // creo la solucion
  DijkstraSolution sol( instance->graph->nodeCount, instance->u );
  // cargo en la solucion, todos los paths del dijkstra desde el nodo inicial
  dijsktra.findPath( instance->graph, &sol );
  // obtengo el path que me interesa
  sol.getPath( instance->v, instance->graph, solution->path, 
                      solution->totalOmega1, solution->totalOmega2 );
}

\end{verbatim}
\normalsize

\subsubsection{Local Search}
\small
\hangindent=0.7cm\begin{verbatim}

int main( int argc, char const* argv[] )
{
  ....
  InitialSolution* initialSolution =  
    heuristicFactory.createInitialSolution( initialSolutionParameter );
  InitialSolution* initialSolutionBestOmega1 =  
    heuristicFactory.createInitialSolution( INITIAL_SOLUTION_A );
  NeighbourhoodSelector* selector = 
    heuristicFactory.createNeighborhoodSelector( neighborhoodSelectorParameter );
  
  // parse the input
  parser.parseInput();  

  for(auto instance:parser.problemInstances)
  {
  Solution* solution = initialSolution->getInitialSolution( instance );
  // si no encuentro el path que cumpla con K, pruebo usando dijkstra con omega1.
  // Tambien puede pasar que no encuentre ningun path, 
  // por lo que totalOmega1 = INF, y en ese 
  // caso tambien pruebo buscar otro path
  if( initialSolution->type != initialSolutionBestOmega1->type && 
    solution->totalOmega1 > instance->K) {
    delete solution;
    solution = initialSolutionBestOmega1->getInitialSolution(instance);
  }
  
  // El dijkstra de omega1 debe cumplir con el K, sino no 
  // tiene sentido correr la heuristica
  if(solution->totalOmega1 <= instance->K) {
    // run the heuristic
    Solution* newSolution = NULL;  
    bool huboMejora = false;
    do
    {
      newSolution = selector->getBestNeighbour( solution );
      // Si no logro mejorar la solucion, termino    
      if(newSolution != NULL) {
           delete solution;
           solution = newSolution;      
           huboMejora = true;  
      } else {            
           huboMejora = false;
      }    
    } while(huboMejora);
  } 
  solution->print();
  delete solution;
  }
  return 0;
}

Solution* NeighbourhoodSelectorA::getBestNeighbour(const Solution* origSolution)
{ 
  // Itero sobre la matriz de soluciones
  // Por cada par de nodos, tengo que buscar si existe un path en la matriz, 
  // tal que sustituyendolo pot path entre el par de nodos de la solucion original 
  // se obtenga una nueva solucion tal que se su totalOmega2 sea menor
  
  // El path de una solution es un vector de ejes   
  // En cada iteracion que encuentro una solucion mejor, voy a cambiar algunos 
  // de los nodos, por lo que tengo que recalcular los nodos de la mejor solucion       
  int edgesCount = origSolution->path.size();
  int nodesCount = edgesCount+1;
  vector<int> nodes(nodesCount); 
  for(int i=0; i<edgesCount; i++) {
    Edge* edge = origSolution->path[i];
    nodes[i] = edge->fromNode;
  }
  // ultimo nodo del path
  nodes[nodesCount-1] = origSolution->path[edgesCount-1]->toNode; 
    
  Solution* bestSolution = NULL;
  for(int i=0; i<nodes.size() - 1; i++) {  
  for(int j=i+1; j<nodes.size(); j++) {       
    // solution con sub path entre los nodos
    Solution* subSolution = origSolution->createSubSolutionBetween(nodes[i], nodes[j]); 
    if(subSolution == NULL) continue; 
    // dijkstra por omega2 entre los nodos
    Solution* solution_ij = getSolvedPathBetween(nodes[i], nodes[j]); 
    if(solution_ij == NULL) continue;
    
    
    // Si el path creado con dijkstra usando omega2, tiene menos omega2 total, 
    // que el path actual entre los nodos i y j, entonces chequeo si al 
    // crear una nueva solucion tendra menos omega2 que la mejor solucion.
    // En ese caso me guardo esta nueva solucion como la mejor hasta ahora.
    // Ademas chequeo que se cumpla con el K requerido.              
    double bestSolutionOmega2 = bestSolution == NULL ? 
                  origSolution->totalOmega2 : 
                  bestSolution->totalOmega2;
    double newSolutionOmega2 = origSolution->totalOmega2 - subSolution->totalOmega2
                   + solution_ij->totalOmega2;
    double newSolutionOmega1 = origSolution->totalOmega1 - subSolution->totalOmega1
                   + solution_ij->totalOmega1;    
    if(newSolutionOmega2 < bestSolutionOmega2 && newSolutionOmega1 <= K) {       
    if(bestSolution != NULL) {
      delete bestSolution;   
    }     
    // creo la nueva mejor solucion
    bestSolution = createNewSolutionReplacingPath(origSolution, solution_ij);      
    }    
    delete subSolution;
  }
  }
  return bestSolution;  
}

Solution* createNewSolutionReplacingPath(const Solution* orig, const Solution* sub) {
  int node1 = sub->path[0]->fromNode;
  int node2 = sub->path[sub->path.size()-1]->toNode;
  Solution* res = new Solution();
  vector<Edge*> edgesToRemove;
  bool subPathStartsAtNode1 = false;
  int subPathStartsAtEdgeIndex = 0;
  
  // primero agrego todos los edges del path original hasta encontrar 
  // alguno de los nodos del subpath
  for(int i=0; i<orig->path.size(); i++) {
    Edge* edge = orig->path[i];
    if(edge->fromNode == node1 || edge->fromNode == node2) {
      subPathStartsAtNode1 = edge->fromNode == node1;
      subPathStartsAtEdgeIndex = i;
      break;
    } else {
      res->path.push_back(edge);
      res->totalOmega1 += edge->omega1;
      res->totalOmega2 += edge->omega2;
    }
  }

  // agrego todos los ejes del sub path
  for(int i=0; i<sub->path.size(); i++) {
    Edge* edge = sub->path[i];
    res->path.push_back(edge);
    res->totalOmega1 += edge->omega1;
    res->totalOmega2 += edge->omega2;
  }

  // busco el nodo desde donde continua el pedazo del path original
  // agrego todos los ejes del path original a partir de ahi
  int fromNode = subPathStartsAtNode1 ? node2 : node1;
  bool addEdges = false;
  for(int i=subPathStartsAtEdgeIndex+1; i<orig->path.size(); i++) {
    Edge* edge = orig->path[i];
    if(edge->fromNode == fromNode) {
      addEdges = true; // encontre el nodo, asi que comienzo a añadir los nodos desde aca
    }
    if(addEdges) {
      res->path.push_back(edge);
      res->totalOmega1 += edge->omega1;
      res->totalOmega2 += edge->omega2;
    }
  }

  return res;
}

\end{verbatim}
\normalsize

\subsubsection{Grasp}

\small
\begin{verbatim}

int main( int argc, char const* argv[] ) 
{
  // instantiate the neighborhood selector using the neighborhood selector parameter
  NeighbourhoodSelector* selector = 
          heuristicFactory.createNeighborhoodSelector( neighborhoodSelectorParameter );
  // parse the input
  parser.parseInput();  

  for(auto instance:parser.problemInstances)
  {
    selector->initialize(instance);    

    // valores arbitrarios basados en n para criterio de terminaciones
    int n = instance->graph->nodeCount;
    int iteracionesSinMejorarCount = 0;
    int iteracionesSinMejorarMax = n;
    int iteracionesMax = n * log(n);
    int iteracionesSinInitialPathCount = 0;
    int iteracionesSinInitialPathMax = n;
    
    Solution* bestSolution = NULL; 
    // aca faltaria hacer un dijkstra por omega1, para ver que al 
    // menos existe una solucion factible

    for(int i = 0; i<iteracionesMax; i++) {  
      // instantiate the initial solution using the initial solution parameter
      InitialSolution* initialSolution =  
              heuristicFactory.createInitialSolution( initialSolutionParameter );    
      Solution* solution = initialSolution->getInitialSolution( instance );
      if(solution->path.size() == 0) {
        // no encontre un path entre u y v
        iteracionesSinInitialPathCount++;
        delete solution;      
        if(iteracionesSinInitialPathCount < iteracionesSinInitialPathMax) {
          continue; // sigo intentando buscar soluciones
        } else {
          break; // me rindo, dejo de buscar soluciones
        }      
      }

      // El dijkstra de omega1 debe cumplir con el K, sino no tiene 
      // sentido correr la heuristica
      if(solution->totalOmega1 <= instance->K) {
        // run the heuristic
        Solution* newSolution = NULL;  
        bool huboMejora = false;
        do
        {
          newSolution = selector->getBestNeighbour( solution );
          // Si no logro mejorar la solucion, termino    
          if(newSolution != NULL) {
            delete solution;
            solution = newSolution;      
            huboMejora = true;  
          } else {            
            huboMejora = false;
          }    
        } while(huboMejora);

        if(bestSolution == NULL) {
          bestSolution = solution;
        } else if(solution->totalOmega2 < bestSolution->totalOmega2) {
          delete bestSolution;
          bestSolution = solution;
        } else {
          delete solution;
          iteracionesSinMejorarCount++;    
        }      
      }     

      if(iteracionesSinMejorarCount > iteracionesSinMejorarMax) {    
        break;
      }    
    }  

    if(bestSolution != NULL) {
      bestSolution->print();
      delete bestSolution;
    } else {
      cout << "no" << endl;
    }  
  }
  
  return 0;
}
\end{verbatim}
\normalsize