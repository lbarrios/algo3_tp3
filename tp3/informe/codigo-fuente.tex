
\subsubsection{Backtracking}

\definecolor{gris}{rgb}{0.3,0.3,0.3}
\definecolor{rojito}{rgb}{1,0,0}
\lstset{language=C++,tabsize=2,basicstyle=\footnotesize,commentstyle=\color{gris},stringstyle=\color{rojito}}

\begin{lstlisting}[caption=backtracking.cpp]
Timer timer( cerr );

int main() {
    int N;
    while (cin >> N && N) {
        BacktrackingHeuristic b;
        b.parseInput(N);
        timer.setInitialTime( "todo_el_codigo" );
        
        if (b.U != b.V) {
            b.initialize();
            vector<Node> adjacent = b.G->getAdjacent(b.U); // se devuelve por referencia
            for (int i = 0; i < adjacent.size(); i++) {
                Node n = adjacent[i];
                Edge *f = b.G->getEdge(b.U, n);
                b.backtrack(f);
            }
        }
        
        timer.setFinalTime( "todo_el_codigo" );
        timer.saveAllTimes();
        b.printSolution();
    }
    return 0;
}
\end{lstlisting}
\begin{lstlisting}[caption=BacktrackingHeuristic::parseInput()]
void BacktrackingHeuristic::parseInput(int N) {
    this->N = N;
    cin >> this->M >> this->U >> this->V >> this->K;
    this->G = new Graph(N);
    this->visited = vector<bool>(N, false);
    int v1, v2;
    double w1, w2;
    for(int i = 0; i < M; i++) {
        cin >> v1 >> v2 >> w1 >> w2;
        this->G->addEdge(v1, v2, w1, w2);
    }
}
\end{lstlisting}
\begin{lstlisting}[caption=BacktrackingHeuristic::initialize()]
void BacktrackingHeuristic::initialize() {
    this->currentBranch.path.push_back(this->U);
    this->bestSolutionFound.totalOmega2 = INFINITE;
    
    DijkstraSolution byOmega1(N, V);
    DijkstraSolution byOmega2(N, V);
    Dijkstra<ObjectiveFunctionA> dijsktra1; 
    Dijkstra<ObjectiveFunctionB> dijsktra2; 
    dijsktra1.findPath(this->G, &byOmega1);
    dijsktra2.findPath(this->G, &byOmega2);
    this->distancesOmega1 = byOmega1.distances;
    this->distancesOmega2 = byOmega2.distances;
}
\end{lstlisting}
\begin{lstlisting}[caption=BacktrackingHeuristic::backtrack()]
void BacktrackingHeuristic::backtrack(Edge *e) {
    Node toNode = e->toNode;
    this->currentBranch.path.push_back(toNode);
    this->currentBranch.totalOmega1 += e->omega1;
    this->currentBranch.totalOmega2 += e->omega2;
    this->visited[toNode] = true;
    
    bool podar = ((this->currentBranch.totalOmega1 + this->distancesOmega1[toNode]) > this->K) ||
        ((this->currentBranch.totalOmega2 + this->distancesOmega2[toNode]) >= this->bestSolutionFound.totalOmega2);

    if (!podar) {
        if (toNode == this->V) {
            this->bestSolutionFound = currentBranch;
        } else {
            vector<Node> adjacent = G->getAdjacent(toNode); // se devuelve por referencia
            for (int i = 0; i < adjacent.size(); i++) {
                Node n = adjacent[i];
                if (!visited[n]) {
                    Edge *f = this->G->getEdge(toNode, n);
                    backtrack(f);
                }
            }
        }
    }

    this->currentBranch.path.pop_back();
    this->currentBranch.totalOmega1 -= e->omega1;
    this->currentBranch.totalOmega2 -= e->omega2;
    this->visited[toNode] = false;
}
\end{lstlisting}
\begin{lstlisting}[caption=BacktrackingHeuristic::printSolution()]
void BacktrackingHeuristic::printSolution() {
    Solution *s = &(this->bestSolutionFound);
    if (s->totalOmega2 == INFINITE) {
        cout << "no" << endl;
        return;
    }

    cout << s->totalOmega1 << " " << s->totalOmega2 << " " << (s->path.size()+1);
    for (int i = 0; i < s->path.size(); i++)
        cout << " " << s->path[i];
    cout << endl;
    return;
}
\end{lstlisting}
\subsubsection{Greedy}

\begin{lstlisting}[caption=greedy\_heuristic\_All.cpp]
#include <iostream>
#include "../common/Timer.h"
#include "GreedyHeuristicAll.h"

using namespace std;

int main(int argc, char* argv[])
{
  Timer timer(cerr);
  GreedyHeuristicAll greedy(&timer);
  greedy.run(); 
    
    return 0;
}
\end{lstlisting}
\begin{lstlisting}[caption=GreedyHeuristicAll.cpp]
#include "GreedyHeuristicAll.h"
#include "../common/Dijkstra.h"
#include "../common/ObjectiveFunctions.h"

GreedyHeuristicAll::GreedyHeuristicAll()
{
  solution = new Solution();
}

GreedyHeuristicAll::GreedyHeuristicAll(Timer* t)
{
  solution = new Solution();
  timer = t;
}
\end{lstlisting}
\begin{lstlisting}[caption=createSolutionA()]
Solution* createSolutionA(ProblemInstance* instance) 
{    
    // creo el dijkstra
    Dijkstra<ObjectiveFunctionA> dijsktra;
    // creo la solucion
    DijkstraSolution dijkstraSolution( instance->graph->nodeCount, instance->u);
    // cargo en la solucion, todos los paths del dijkstra desde el nodo inicial
    dijsktra.findPath( instance->graph, &dijkstraSolution );
    // obtengo el path que me interesa
    Solution* solution = new Solution();    
    dijkstraSolution.getPath( instance->v, instance->graph, solution->path, solution->totalOmega1, solution->totalOmega2 );
    return solution;
}
\end{lstlisting}
\begin{lstlisting}[caption=createSolutionB()]
Solution* createSolutionB(ProblemInstance* instance) 
{
    // creo el dijkstra    
    Dijkstra<ObjectiveFunctionB> dijsktra;
    // creo la solucion
    DijkstraSolution dijkstraSolution( instance->graph->nodeCount, instance->u);
    // cargo en la solucion, todos los paths del dijkstra desde el nodo inicial
    dijsktra.findPath( instance->graph, &dijkstraSolution );
    // obtengo el path que me interesa
    Solution* solution = new Solution();    
    dijkstraSolution.getPath( instance->v, instance->graph, solution->path, solution->totalOmega1, solution->totalOmega2 );
    return solution;
}
\end{lstlisting}
\begin{lstlisting}[caption=createSolutionC()]
Solution* createSolutionC(ProblemInstance* instance) 
{
    // creo el dijkstra    
    Dijkstra<ObjectiveFunctionC> dijsktra;
    // creo la solucion
    DijkstraSolution dijkstraSolution( instance->graph->nodeCount, instance->u);
    // cargo en la solucion, todos los paths del dijkstra desde el nodo inicial
    dijsktra.findPath( instance->graph, &dijkstraSolution );
    // obtengo el path que me interesa
    Solution* solution = new Solution();    
    dijkstraSolution.getPath( instance->v, instance->graph, solution->path, solution->totalOmega1, solution->totalOmega2 );
    return solution;
}
\end{lstlisting}
\begin{lstlisting}[caption=getBestSolution()]
Solution* getBestSolution(ProblemInstance* instance) 
{
    Solution* solutionB = createSolutionB(instance);    
    if(solutionB->totalOmega1 <= instance->K) {
        return solutionB;
    }
    // si la solucion no es factible probamos con otras funciones objetivo    
    Solution* solutionA = createSolutionA(instance);
    Solution* solutionC = createSolutionC(instance);
    if(solutionC->totalOmega1 <= instance->K) {
        if(solutionA->totalOmega2 < solutionC->totalOmega2) {            
            return solutionA;
        }         
        return solutionC;
    }    
        
    if(solutionA->totalOmega1 <= instance->K) {
        return solutionA;
    }
    return NULL;   
}
\end{lstlisting}
\begin{lstlisting}[caption=GreedyHeuristicAll::resolveInstance()]
void GreedyHeuristicAll::resolveInstance( ProblemInstance* instance ){
  solution = getBestSolution(instance);
}
\end{lstlisting}
\begin{lstlisting}[caption=GreedyHeuristicAll::run()]
void GreedyHeuristicAll::run()
{
  Parser parser;
  parser.parseInput();

  for ( int i = 0; i < parser.problemInstances.size(); i++ )
  {
    ProblemInstance* instance = parser.problemInstances[i];
    
    timer->setInitialTime("todo_el_codigo");
    resolveInstance( instance );       
    timer->setFinalTime("todo_el_codigo");
    timer->saveAllTimes();
    
    if(!solution) {
      cout << "no" << endl;
    } else{
      solution->print();
    }    
  }
}
\end{lstlisting}

\subsubsection{Local Search}

\begin{lstlisting}[caption=local\_search.cpp]
#include "../common/Graph.h"
#include "../common/Parser.h"
#include "../common/DijkstraSolution.h"
#include "../common/Dijkstra.h"
#include "../common/GreedyHeuristic.h"
#include "../common/Solution.h"
#include "../common/Parser.h"
#include "../common/Timer.h"
#include "NeighbourhoodSelectorA.h"
#include "InitialSolution.h"

Parser parser;
Timer timer( cerr );


int main( int argc, char const* argv[] )
{
  /*****************
    Initialization
  ******************/
  // instantiate the initial solution using the initial solution parameter
  InitialSolution* initialSolution = new InitialSolution();
  // instantiate the neighborhood selector using the neighborhood selector parameter
  NeighbourhoodSelector* selector = new NeighbourhoodSelectorA();
  // parse the input
  parser.parseInput();

  /*************************
    Iterate over instances
  *******************'*******/
  for(auto instance:parser.problemInstances)
  {
    /*************
      Resolution
    **************/
    selector->initialize(instance);

    // obtain the initial time
    timer.setInitialTime( "todo_el_codigo" );
    // obtain the initial solution

    int initialSolutionOmega2;

    Solution* solution = initialSolution->getInitialSolution( instance );

    // El dijkstra de omega1 debe cumplir con el K, sino no tiene sentido correr la heuristica
    // si solution no es valida entonces, entonces es NULL
    if(solution != NULL) {
      initialSolutionOmega2 = solution->totalOmega2;

      // run the heuristic
      Solution* newSolution = NULL;
      bool huboMejora = false;
      do
      {
        newSolution = selector->getBestNeighbour( solution );
        // Si no logro mejorar la solucion, termino
        if(newSolution != NULL) {
          delete solution;
          solution = newSolution;          
          huboMejora = true;  
        } else {                      
          huboMejora = false;
        }
      } while(huboMejora);
    } 

    // obtain the final time
    timer.setFinalTime( "todo_el_codigo" );

    /***************
      Output Print
    ****************/
    // print the solution
    if(solution) {
      solution->print();
      delete solution;
    }
    else
    {
      cout << 0 << endl;
    }

    // save all obtained times to output
    timer.saveAllTimes();
  }
  return 0;
}
\end{lstlisting}
\begin{lstlisting}[caption=createSolutionA()]
Solution* createSolutionA(ProblemInstance* instance) 
{    
    // creo el dijkstra
    Dijkstra<ObjectiveFunctionA> dijsktra;
    // creo la solucion
    DijkstraSolution dijkstraSolution( instance->graph->nodeCount, instance->u);
    // cargo en la solucion, todos los paths del dijkstra desde el nodo inicial
    dijsktra.findPath( instance->graph, &dijkstraSolution );
    // obtengo el path que me interesa
    Solution* solution = new Solution();    
    dijkstraSolution.getPath( instance->v, instance->graph, solution->path, solution->totalOmega1, solution->totalOmega2 );
    return solution;
}
\end{lstlisting}
\begin{lstlisting}[caption=createSolutionB()]
Solution* createSolutionB(ProblemInstance* instance) 
{
    // creo el dijkstra    
    Dijkstra<ObjectiveFunctionB> dijsktra;
    // creo la solucion
    DijkstraSolution dijkstraSolution( instance->graph->nodeCount, instance->u);
    // cargo en la solucion, todos los paths del dijkstra desde el nodo inicial
    dijsktra.findPath( instance->graph, &dijkstraSolution );
    // obtengo el path que me interesa
    Solution* solution = new Solution();    
    dijkstraSolution.getPath( instance->v, instance->graph, solution->path, solution->totalOmega1, solution->totalOmega2 );
    return solution;
}
\end{lstlisting}
\begin{lstlisting}[caption=createSolutionC()]
Solution* createSolutionC(ProblemInstance* instance) 
{
    // creo el dijkstra    
    Dijkstra<ObjectiveFunctionC> dijsktra;
    // creo la solucion
    DijkstraSolution dijkstraSolution( instance->graph->nodeCount, instance->u);
    // cargo en la solucion, todos los paths del dijkstra desde el nodo inicial
    dijsktra.findPath( instance->graph, &dijkstraSolution );
    // obtengo el path que me interesa
    Solution* solution = new Solution();    
    dijkstraSolution.getPath( instance->v, instance->graph, solution->path, solution->totalOmega1, solution->totalOmega2 );
    return solution;
}
\end{lstlisting}
\begin{lstlisting}[caption=InitialSolution::getInitialSolution()]
Solution* InitialSolution::getInitialSolution(ProblemInstance* instance)
{   
    Solution* solutionB = createSolutionB(instance);    
    if(solutionB->totalOmega1 <= instance->K) {
        return solutionB;
    }
    // si la solucion no es factible probamos con otras funciones objetivo    
    Solution* solutionA = createSolutionA(instance);
    Solution* solutionC = createSolutionC(instance);
    if(solutionC->totalOmega1 <= instance->K) {
        if(solutionA->totalOmega2 < solutionC->totalOmega2) {
            return solutionA;
        } 
        return solutionC;
    }
    if(solutionA->totalOmega1 <= instance->K) {
        return solutionA;
    }
    return NULL;    
}
\end{lstlisting}
\begin{lstlisting}[caption=NeighbourhoodSelectorA::removeCycles()]
Solution* NeighbourhoodSelectorA::removeCycles(Solution* solution) 
{
  int* nextNodes = new int[nodeCount];
  for(int i=0; i<nodeCount; i++) {
    nextNodes[i] = 0;
  }  
  for(unsigned int i=0; i<solution->path.size(); i++) {
    Edge* edge = solution->path[i];    
    nextNodes[edge->fromNode] = edge->toNode;        
  }        

  vector<int> newPath;
  int firstNode = solution->path[0]->fromNode;
  int lastNode = solution->path[solution->path.size()-1]->toNode;  
  int next = firstNode;
  newPath.push_back(firstNode);     
  int a = 0;
  while(nextNodes[next] != lastNode)
  {    
    next = nextNodes[next];        
    newPath.push_back(next);    
  }

  newPath.push_back(lastNode);
  //cout << lastNode << endl;
  
  Solution* newSolution = new Solution();
  for(int i=0; i<newPath.size()-1; i++){    
    Edge* edge = solution->getEdgeBetween(newPath[i], newPath[i+1]);    
    Edge* newEdge = new Edge(edge->fromNode, edge->toNode, edge->omega1, edge->omega2);
    newSolution->path.push_back(newEdge);
    newSolution->totalOmega1 += newEdge->omega1;
    newSolution->totalOmega2 += newEdge->omega2;
  }

  return newSolution;
}
\end{lstlisting}
\begin{lstlisting}[caption=createNewSolutionReplacingPath()]
Solution* createNewSolutionReplacingPath(const Solution* orig, const Solution* sub) {
  int node1 = sub->path[0]->fromNode;
  int node2 = sub->path[sub->path.size()-1]->toNode;
  Solution* res = new Solution();
  vector<Edge*> edgesToRemove;
  bool subPathStartsAtNode1 = false;
  int subPathStartsAtEdgeIndex = 0;
  // primero agrego todos los edges del path original hasta encontrar alguno de los nodos del subpath
  for(int i=0; i<orig->path.size(); i++) {
    Edge* edge = orig->path[i];
    if(edge->fromNode == node1 || edge->fromNode == node2) {
      subPathStartsAtNode1 = edge->fromNode == node1;
      subPathStartsAtEdgeIndex = i;
      break;
    } else {
      res->path.push_back(edge);
      res->totalOmega1 += edge->omega1;
      res->totalOmega2 += edge->omega2;
    }
  }

  // agrego todos los ejes del sub path
  for(int i=0; i<sub->path.size(); i++) {
    Edge* edge = sub->path[i];
    res->path.push_back(edge);
    res->totalOmega1 += edge->omega1;
    res->totalOmega2 += edge->omega2;
  }

  // busco el nodo desde donde continua el pedazo del path original
  // agrego todos los ejes del path original a partir de ahi
  int fromNode = subPathStartsAtNode1 ? node2 : node1;
  bool addEdges = false;
  for(int i=subPathStartsAtEdgeIndex+1; i<orig->path.size(); i++) {
    Edge* edge = orig->path[i];
    if(edge->fromNode == fromNode) {
      addEdges = true; // encontre el nodo, asi que comienzo a aniadir los nodos desde aca
    }
    if(addEdges) {
      res->path.push_back(edge);
      res->totalOmega1 += edge->omega1;
      res->totalOmega2 += edge->omega2;
    }
  }

  return res;
}
\end{lstlisting}
\begin{lstlisting}[caption=NeighbourhoodSelectorA::getBestNeighbour()]
Solution* NeighbourhoodSelectorA::getBestNeighbour(const Solution* origSolution)
{ 
  // Itero sobre la matriz de soluciones
  // Por cada par de nodos, tengo que buscar si existe un path en la matriz, 
  // tal que sustituyendolo pot path entre el par de nodos de la solucion original 
  // se obtenga una nueva solucion tal que se su totalOmega2 sea menor
  
  // Nota: Habiamos pensado en usar DeltaOmega2, pero si K es muy grande,
  // nos sobraria mucho K que podriamos haber usado para sustituir en cada paso
  // por un path con minimo omega2                
  
  // El path de una solution es un vector de ejes   
  // En cada iteracion que encuentro una solucion mejor, voy a cambiar algunos de los nodos, 
  // por lo que tengo que recalcular los nodos de la mejor solucion             
  int edgesCount = origSolution->path.size();
  int nodesCount = edgesCount+1;
  vector<int> nodes(nodesCount); 
  for(int i=0; i<edgesCount; i++) {
    Edge* edge = origSolution->path[i];
    nodes[i] = edge->fromNode;
  }
  nodes[nodesCount-1] = origSolution->path[edgesCount-1]->toNode; // ultimo nodo del path
      
  Solution* bestSolution = NULL;  
  for(int i=0; i<nodes.size() - 1; i++) {    
    for(int j=i+1; j<nodes.size(); j++) {           
      Solution* subSolution = origSolution->createSubSolutionBetween(nodes[i], nodes[j]); // solution con sub path entre los nodos            
      Solution* solution_ij = getSolvedPathBetween(nodes[i], nodes[j]); // dijkstra por omega2 entre los nodos
      if(solution_ij == NULL) {
        continue;
      }      
      
      // Si el path creado con dijkstra usando omega2, tiene menos omega2 total, que el path actual entre
      // los nodos i y j, entonces chequeo si al crear una nueva solucion tendra menos omega2 que la mejor solucion.
      // En ese caso me guardo esta nueva solucion como la mejor hasta ahora.
      // Ademas chequeo que se cumpla con el K requerido.                          
      double bestSolutionOmega2 = bestSolution == NULL ? origSolution->totalOmega2 : bestSolution->totalOmega2;
      double newSolutionOmega2 = origSolution->totalOmega2 - subSolution->totalOmega2 + solution_ij->totalOmega2;
      double newSolutionOmega1 = origSolution->totalOmega1 - subSolution->totalOmega1 + solution_ij->totalOmega1;            
      if(newSolutionOmega2 < bestSolutionOmega2 && newSolutionOmega1 <= K) {                     
        if(bestSolution != NULL) {
          delete bestSolution;   
        }       
        // creo la nueva mejor solucion        
        bestSolution = createNewSolutionReplacingPath(origSolution, solution_ij);           
        Solution* bestSolutionWithoutCycles = removeCycles(bestSolution);                         
        delete bestSolution;
        bestSolution = bestSolutionWithoutCycles;        
      }      
      delete subSolution;
    }
  }
    
  return bestSolution;  
}
\end{lstlisting}
\begin{lstlisting}[caption=NeighbourhoodSelector::initialize()]
void NeighbourhoodSelector::initialize(ProblemInstance* instance)
{	
	deleteMatrix();
	nodeCount = instance->graph->nodeCount;
	pathMatrix = new Solution*[nodeCount*nodeCount];     
  K = instance->K;   

  // Como los paths entre j y k son iguales a los paths entre k y j, pero al reves
  // primero encuentro los paths entre j y k y luego los doy vuelta
	for(int j=1; j<=nodeCount; j++) {
  	Dijkstra<ObjectiveFunctionOmega2> dijsktra;          
    DijkstraSolution dijkstraSolution( instance->graph->nodeCount, j );                
    dijsktra.findPath( instance->graph, &dijkstraSolution );                      
    for(int k=1; k<=nodeCount; k++) {                    
      if(j==k) continue;
    	  Solution* solution = new Solution();        
    	  dijkstraSolution.getPath(k, instance->graph, solution->path, solution->totalOmega1, solution->totalOmega2);          
        // si no hay ninun path entre los nodos j y k entonces, cargo NULL en la matrix             
        pathMatrix[j-1 + (k-1) * nodeCount] = solution->path.size() > 0 ? solution : NULL;                                  
    }
	}
}
\end{lstlisting}
\begin{lstlisting}[caption=NeighbourhoodSelector::getSolvedPathBetween()]
Solution* NeighbourhoodSelector::getSolvedPathBetween(int node1, int node2) {  
  return pathMatrix[node1-1 + (node2-1) * nodeCount];
}
\end{lstlisting}
\subsubsection{Grasp}


\begin{lstlisting}[caption=grasp.cpp]
#include "../common/Graph.h"
#include "../common/Parser.h"
#include "../common/DijkstraSolution.h"
#include "../common/Dijkstra.h"
#include "../common/DijkstraRandomized.h"
#include "../common/GreedyHeuristic.h"
#include "../common/Solution.h"
#include "../common/Parser.h"
#include "../common/Timer.h"
#include "NeighbourhoodSelectorA.h"
#include "InitialSolution.h"
#include <math.h>

Parser parser;
Timer timer( cerr );


int main( int argc, char const* argv[] ) 
{
    /* Semilla random */
    srand (time(NULL));

    /*****************
      Initialization
    ******************/
            
    // parse the input
    parser.parseInput();  

    // obtain the initial time
    timer.setInitialTime( "todo_el_codigo" );
    // obtain the initial solution

    /*************************
      Iterate over instances
    **************************/
    for(auto instance:parser.problemInstances)
    {        
        /*************
         Resolution
        **************/           
        NeighbourhoodSelector* selector = new NeighbourhoodSelectorA();
        selector->initialize(instance);                        

        // valores arbitrarios basados en n para criterio de terminaciones
        int n = instance->graph->nodeCount;
        int iteracionesSinMejorarCount = 0;
        int iteracionesSinMejorarMax = n*n/4;
        int iteracionesMax = n*n/2;
        int iteracionesSinInitialPathCount = 0;
        int iteracionesSinInitialPathMax = n*n/4;
        
        Solution* bestSolution = NULL;            

        for(int i = 0; i<iteracionesMax; i++) {              
            // instantiate the initial solution using the initial solution parameter
            InitialSolution* initialSolution = new InitialSolution();              
            Solution* solution = initialSolution->getInitialSolution( instance );                     
            if(solution->path.size() == 0) {
                // no encontre un path entre u y v
                iteracionesSinInitialPathCount++;
                delete solution;            
                if(iteracionesSinInitialPathCount < iteracionesSinInitialPathMax) {
                    continue; // sigo intentando buscar soluciones
                } else {
                    break; // me rindo, dejo de buscar soluciones
                }            
            }                        

            // El dijkstra de omega1 debe cumplir con el K, sino no tiene sentido correr la heuristica            
            if(solution->totalOmega1 <= instance->K) {                               
                // run the heuristic
                Solution* newSolution = NULL;    
                bool huboMejora = false;
                do
                {                                        
                    newSolution = selector->getBestNeighbour( solution );                                                            
                    // Si no logro mejorar la solucion, termino      
                    if(newSolution != NULL) {                        
                        //cout << newSolution->totalOmega2 << endl; 
                        delete solution;
                        solution = newSolution;          
                        huboMejora = true;                          
                    } else {                      
                        huboMejora = false;
                    }        
                } while(huboMejora);

                if(bestSolution == NULL) {
                    bestSolution = solution;
                } else if(solution->totalOmega2 < bestSolution->totalOmega2) {
                    delete bestSolution;
                    bestSolution = solution;
                } else {
                    delete solution;
                    iteracionesSinMejorarCount++;      
                }            
            }       

            if(iteracionesSinMejorarCount > iteracionesSinMejorarMax) {      
                break;
            }
        }  

        // obtain the final time
        timer.setFinalTime( "todo_el_codigo" );

        /***************
          Output Print
        ****************/
        // print the solution
        if(bestSolution != NULL) {
            bestSolution->print();
            delete bestSolution;
        } else {
            cout << "no" << endl;
        }    
        
        // save all obtained times to output
        timer.saveAllTimes();
    }    
    
    return 0;
}
\end{lstlisting}
\begin{lstlisting}[caption=InitialSolution::getInitialSolution()]
Solution* InitialSolution::getInitialSolution(ProblemInstance* instance)
{    
  // creo el dijkstra
    DijkstraRandomized<ObjectiveFunctionA> dijsktraRandom;
    // creo la solucion
    DijkstraSolution dijkstraSolution( instance->graph->nodeCount, instance->u);
    // cargo en la solucion, todos los paths del dijkstra desde el nodo inicial
    dijsktraRandom.findPath( instance->graph, &dijkstraSolution );
    // obtengo el path que me interesa
    Solution* solution = new Solution();    
    dijkstraSolution.getPath( instance->v, instance->graph, solution->path, solution->totalOmega1, solution->totalOmega2 );
    return solution;
}
\end{lstlisting}