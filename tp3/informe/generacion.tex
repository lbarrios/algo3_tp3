Se generaron dos familias principales de instancias de CACM:

\begin{itemize}
    \item Instancia \textit{`aleatoria'}
El objetivo fue generar una instancia de la que no se pueda suponer nada, para evitar cualquier tipo de sesgo.

Se tomó un valor fijo de cota para todos los parametros, llamado $C$.

\begin{itemize}
    \item $n$ se eligió uniformemente entre $\left\{{1, ..., C}\right\}$.
    \item $k$ se eligió uniformemente entre $\left\{{1, ..., C}\right\}$.
    \item $u$ se eligió uniformemente entre $\left\{{1, ..., n}\right\}$.
    \item $v$ se eligió uniformemente entre $\left\{{1, ..., n}\right\}$.
    \item $m$ se eligió uniformemente entre $\left\{{0, ..., \frac{n(n-1)}{2}}\right\}$.
    \item Para cada arista se eligieron como extremos dos enteros uniformente entre $\left\{{1, ..., n}\right\}$ (si los nodos correspondientes ya eran adyacentes, se vuelve a elegir). Los pesos según $\omega_1$ y $\omega_2$ se eligieron como valores de punto flotante uniformemente en el intervalo $(0, C]$.
\end{itemize}
Observación: $C$ depende del algoritmo para el que haya sido generada la instancia. Por ejemplo, para el \textit{backtracking} no tiene sentido usar $C > 30$, pues el tiempo de ejecución puede llegar a ser demasiado grande.

    \item Instancia \textit{`mágica'}

Estas instancias fueron generadas para la medición de calidad de las heurísticas. Se conoce de antemano el valor de la solución óptima, lo que en instancias generales solo se podría saber corriendo un algoritmo exacto, que puede tardar mucho en grafos grandes. Además se pretende que el peso de las aristas esté \textit{balanceado}, es decir, a pesos bajos por $\omega_1$ correspondan pesos altos por $\omega_2$, y a la inversa.

\begin{itemize}
    \item $n$ es elegida como parámetro.
    \item $k$ vale $n-1$.
    \item $u$ vale 1.
    \item $v$ vale $n$.
    \item $m$ se eligió uniformemente entre $\left\{{n-1, ..., \frac{n(n-1)}{2}}\right\}$.
    \item primero se agregan las siguientes $n-1$ aristas: (1, 2), (2, 3), ..., ($n-1$, $n$) formando un camino hamiltoniano entre $u$ y $v$. Los pesos de estas aristas, según tanto $\omega_1$ como $\omega_2$, van a ser 1. De esta forma generamos un camino $I$ válido, ya que $\omega_1(I)$ vale $n-1 \leq k$.
    \item Para las aristas que quedan se eligen como extremos dos enteros $a$, $b$ uniformente entre $\left\{{1, ..., n}\right\}$ (si los nodos correspondientes ya eran adyacentes, se vuelve a elegir). El peso va a guardar relación con $|b-a|$, valor que denotaremos \textit{diff}.
\end{itemize}
\end{itemize}

El peso según $\omega_1$ va a ser generado de manera uniforme entre $\left\{{0, ..., 2*diff}\right\}$. De esta forma la esperanza del peso va a ser \textit{diff}. El peso según $\omega_2$ va a valer 2*\textit{diff} menos el peso según $\omega_2$. De esta forma la esperanza de este peso también va a valer \textit{diff}. Así quedan balanceados los dos pesos simétricamente alrededor de \textit{diff}. Por ejemplo, si tenemos la arista (1, 9), \textit{diff} vale 8. Si $\omega_1$ resulta 6, entonces $\omega_2$ lo fijamos en 10.

Sea $P$ un camino, entonces la esperanza de su peso - tanto por $\omega_1$ como por $\omega_2$ - es la sumatoria de las esperanzas de las aristas que lo constituyen. Es la sumatoria del \textit{diff} de cada arista. Los caminos que nos interesan van de $u$ a $v$. Cualquier camino de 1 a $n$  va a tener una suma de \textit{diff} mayor o igual a $n-1$, ya que por cada unidad que uno avanza entre 1 y $n$ utilizando una arista, la arista suma 1 a su \textit{diff}. Como se deben avanzar $n-1$ unidades, entonces la sumatoria de los \textit{diff} va a ser mayor o igual a $n-1$.

Lo importante es que el centro de simetría de los pesos de $P$ - la sumatoria de los \textit{diff} - va a ser mayor o igual que $n-1$. Luego, si $\omega_2(P)$ $<$ $n-1$, entonces $\omega_1(P)$ = 2 * sumatoria(\textit{diff}) - $\omega_2(P)$ $>$ $n-1$. Por lo tanto no sería un camino válido pues se excedería del valor de $k$. Entonces el camino óptimo tiene $\omega_2$ $\geq$ $n-1$. El camino $I$ (válido) definido previamente es, entonces, uno de los caminos óptimos. En la familia de grafos \textit{magica} sabemos luego que el camino óptimo tiene $\omega_1$ = $n-1$.

Otra ventaja de generar grafos de esta forma es que permite una gran cantidad de caminos alternativos para llegar de $u$ a $v$. Se espera que estos nodos sean los más alejados entre si en todo el grafo, ya que para llegar de 1 a $n$ las aristas del camino van a tener una sumatoria de \textit{diff} - que es la esperanza de la longitud del camino - de por lo menos $n-1$. La esperanza de distancia para cualquier camino entre el nodo 1 y el nodo $n/2$, por ejemplo, es mayor a $n/2 - 1$.

Si uno forma un camino entre 1 y $n$ como secuencia creciente de vértices, éste tiene esperanza de peso $n-1$. Por ejemplo:
\begin{itemize}
    \item 1 $\rightarrow$ n     \hspace{2cm}La esperanza de peso es $n-1$
    \item 1 $\rightarrow$ n/2 $\rightarrow$ n   \hspace{0.5cm}     La esperanza de peso es $(n/2 -1) + (n/2) = n-1$
    \item 1 $\rightarrow$ 7 $\rightarrow$ n    \hspace{1cm}    La esperanza de peso es $(6 + (n - 7) = n-1$
\end{itemize}

Entonces todos las secuencias crecientes de vertices que empiezan en 1 y terminan en $n$ van a ser candidatos a ser el camino óptimo, ya que todos tienen la esperanza de peso - tanto por $\omega_1$ como por $\omega_2$ - alrededor de $n-1$, que es el valor de $k$. Esto permite un análisis más interesante de la calidad de las distintas heurísticas.

\newpage
Para cada familia de instancias, se consideraron tres subfamilias:

\begin{itemize}
    \item \textit{Densidad baja}

    La cantidad de aristas se definió de antemano como $3n$, es decir, $O(n)$.
    \item \textit{Densidad media}

    La cantidad de aristas se definió de antemano como $n\sqrt{n}$, es decir, $O(n\sqrt{n})$.
    \item \textit{Densidad alta}

    La cantidad de aristas se definió de antemano como la máxima: $\frac{n(n-1)}{2}$ es decir, $O(n^2)$.
\end{itemize}
